\section{Revisione dei Requisiti}
\label{sec:revisione_requisiti}

\begin{enumerate}
\item Il sistema deve offrire all'Utente la funzionalità di registrarsi
\item Di ogni Utente si vuole memorizzare nome, cognome, email e affiliazione
\item Il sistema deve offrire all'Utente la possibilità di accedere
\item Il sistema deve assegnare un ID univoco ad ogni utente
\item Il sistema deve dare la possibilità all'Utente di scegliere il proprio ruolo al momento della registrazione
\item Il sistema deve dare ad un Organizzatore la possibilità di creare una nuova conferenza
\item Di ogni conferenza si vuole memorizzare il titolo, una descrizione ed una singola scadenza
  per la chiusura delle sottomissioni
\item Il sistema deve dare ad un Organizzatore la possibilità di gestire una o più conferenze
\item Il sistema deve dare la possibilità ad un Organizzatore di monitorare le proprie conferenze
\item Il sistema deve date la possibilità ad un Organizzatore di accedere agli articoli legati alle proprie conferenze
\item Il sistema deve dare la possibilità ad un Autore di visualizzare le conferenze attive
\item Il sistema deve dare la possibilità ad un Autore di sottomettere un articolo ad una conferenza attiva
\item Di ogni Articolo si vuole memorizzare il titolo, l'abstract, co-autori
\item L'abstract di un Articolo può essere composto al più da 250 caratteri
\item Un Articolo può avere un massimo di 3 co-autori, quindi un massimo di 4 autori totali
\item I co-autori per un Articolo devono essere scelti dalla lista degli autori già registrati
\item Il sistema deve impostare lo stato di un articolo sottomesso ad una conferenza da un Autore a ``sottomesso''
\item Il sistema deve dare la possibilità ad un Autore di visualizzare i propri articoli
\item Il sistema deve dare la possibilità ad un Organizzatore di assegnare tre revisori ad ogni articolo di una conferenza dopo la data di scadenza
\item Il sistema deve garantire la mancanza di conflitti di interesse tra i revisori e gli articoli a loro assegnati
\item Il sistema deve cambiare lo stato di un articolo da ``sottomesso'' a ``in revisione'' una volta che sono stati assegnati i tre revisori
\item Il sistema deve dare la possibilità ad un revisore di inserire l'esito della propria revisione di un articolo
\item Dell'esito di una revisione si vuole memorizzare il commento, l'esito ed il punteggio
\item Il sistema deve dare la possibilità ad un Organizzatore di consultare lo stato aggregato di una conferenza, ossia numero totale di articoli sottomessi
  , numero di articoli in revisione e stato di ogni articolo
\item Il sistema deve garantire una gestione rigorosa dei permessi e delle visibilità
\item Il sistema deve garantire l'anonimato dei revisore nel processo di peer review
\item Il sistema deve prevedere notifiche automatiche via email per le principali scadenze
\item Il sistema deve offrire interfacce grafiche separeate per autori e organizzatori
\item Il sistema deve essere accessibile da dispositivi desktop
\item Il sistema deve essere accessibile da dispositivi mobili
\end{enumerate}

%%% Local Variables:
%%% mode: LaTeX
%%% TeX-master: "main"
%%% End:
