\subsection{Scenari}
% TODO: Rifare i scenari
\label{sec:scenari}

\subsubsection{Scenario Registrazione}
\label{sec:scenari:registrazione}
% DONE: Rifare scenario Registrazione
\begin{tabular}{|p{3cm}|p{7cm}|}
\hline 
\rowcolor{Orchid}
Caso d'uso & Registrazione \\
\hline
Attore Primario & Utente\\
\hline
Attore Secondario & \\
\hline
Descrizione & Permette la registrazione di un utente\\
\hline
  Sequenza Eventi &
  \begin{enumerate}
    \item L'utente inserisce Nome, Cognome, Email, Password, affiliazione
    \item L'utente sceglie il proprio ruolo
    \item L'utente clicca sul bottone registrarsi
    \item Il sistema verifica la validità dell'indirizzo email
    \item IF l'indirizzo email è valido
   \begin{enumerate}
    \item il sistema controlla che l'indirizzo non sia già presente
     \item IF indirizzo email non è presente
       \begin{enumerate}
         \item il sistema controlla la validità della password
           \begin{enumerate}
            \item IF la password è valida
               \begin{enumerate}
               \item Il sistema salva i dati dell'Utente
               \item punto di estensione: GestionePermessiVisibilià
               \end{enumerate}
            \item ALTRIMENTI il Sistema mostra un messaggio di errore
          \end{enumerate}
        \item ALTRIMENTI il Sistema mostra un messaggio di errore spiegando che non è possibile usare l'email
       \end{enumerate}         
      \end{enumerate}
 \item ALTRIMENTI il Sistema mostra un messaggio di errore spiegando che l'email non è valida
  \end{enumerate}\\
\hline
Post-Condizioni: & Il cliente è registrato nel sistema \\
  \hline
\end{tabular}\\ 
\begin{tabular}{|p{3cm}|p{7cm}|}
  \hline
  Sequenza degli eventi alternativa & \begin{itemize}
  \item Se nel punto 5 l'indirizzo email non è valido viene mostrato un messaggio di errore all'utente e gli si chiede di inserire un nuovo indirizzo email
\item Se nel punto 5.2 l'indirizzo email è già presente nel sistema si mostra un messaggio di errore e si chiede all'utente di usare credenziali diverse
\item Se nel punto 5.2.1.1 la password risulta non valida viene mostrato un messaggio di errore all'utente e gli viene chiesto di inserire una nuova password
  \end{itemize}\\
\hline
\end{tabular} \\

\subsubsection{Scenario Accesso}
% DONE: Rifare scenario Accesso
\begin{tabular}{|p{3cm}|p{7cm}|}
\hline 
\rowcolor{Orchid}
Caso d'uso & Accesso \\
\hline
Attore Primario & Utente\\
\hline
Attore Secondario & \\
\hline
Descrizione & Permette l'accesso di un utente al suo account\\
\hline
Pre-Condizioni& L'utente sia già registrato\\
\hline
  Sequenza Eventi&
\begin{enumerate}
  \item L'utente inserisce email e password
\item Il sistema controlla la validità dell'email
\item IF l'email inserita è valida
\begin{enumerate}
  \item Il sistema controlla se l'email è presente nel sistema
  \item IF l'email è presente nel sistema
\begin{enumerate}
  \item il sistema controlla se le credenziali sono corrette
\item IF le credenziali sono corrette
\begin{enumerate}
  \item punto di estensione: GestionePermessiVisibilità
\item IF la GestionePermessiVisibilità va a buon fine
\begin{enumerate}
  \item mostra all'utente la giusta schermata per il suo ruolo
\end{enumerate}
\item ALTRIMENTI mostra un messaggio di errore
\end{enumerate}
\item ALTRIMENTI mostra un messaggio di errore e chiede
all'utente nuove credenziali
\end{enumerate}
\item ALTRIMENTI mostra un messaggio di errore e
chiede all'utente di inserire nuove credenziali
\end{enumerate}
\item ALTRIMENTI mostra un messaggio di errore e
chiede all'utente di inserire nuove credenziali
\end{enumerate}\\
\hline
Post-Condizioni: &il cliente ha l'accesso al suo account \\
  \hline
\end{tabular}\\
\begin{tabular}{|p{3cm}|p{7cm}|}
\hline
  Sequenza degli eventi alternativa & \begin{itemize}
  \item Nel punto 3 se l'email inserita non è valida viene mostrato un messaggio di
errore e viene chiesto all'utente di inserire un nuovo indirizzo email
\item Nel punto 3.2 se l'indirizzo email non è presente nel sistema viene mostrato un
messaggio di errore e viene chiesto all'utente di inserire delle credenziali corrette
\item Nel punto 3.2.2 se le credenziali non sono corrette viene mostrato un messaggio
di errore e viene chiesto all'utente di inserire delle credenziali corrette
\item Nel punto 3.2.2.2 se la GestioneVisibilitàPermessi non va a buon fine
viene mostrato un messaggio di errore
\end{itemize}\\
\hline
\end{tabular}\\

\subsubsection{Scenario Gestione Permessi di Visibilità}
\begin{tabular}{|p{3cm}|p{7cm}|}
\hline 
\rowcolor{Orchid}
Caso d'uso & Scenario Gestione Permessi di Visibilità \\
\hline
Attore Primario & Sistema\\
\hline
Attore Secondario & \\
\hline
Descrizione & Il sistama va a far visualizzare la sezione dedicata all'utente in base al ruolo occupato\\
\hline
 Sequenza Eventi & 
\begin{enumerate}
  \item L'untente va ad inserire le credenziali di accesso
  \item Il sitema va a verificare la validità delle credenziali inserite 
  \item IF le credenziali inserite risultino valide
  \begin{enumerate}
    \item L'utente accede al sistema
    \item IF il ruolo occupato dall'utente è organizzatore
    \begin{enumerate}
      \item Il sitema va a far visualizzare la sezione dedicata agli organizzatori
    \end{enumerate}
  \item ALTRIMENTI il sistema va a visualizzare la sezione dedicata agli autori
  \end{enumerate}
\item ALTRIMENTI il sistema mopstra a schermo un messaggio di errore e chiede il rinserimento delle crdenziali di accesso
\end{enumerate}\\
\hline
Post Condizioni: & L'utente va la a visualizare la sezzione dedicata in base al ruolo occupato\\
\hline
Sequenza degli eventi alternativa & \begin{itemize}
  \item Nel punto 3 se le credenziali non risultinano valide il sistema va a mostrare a schermo un messaggio di errore e chiede il rinserimento delle credenziali
  \item Nel punto 3.2 il ruolo dell'utente non risulta organizzatore il sistema va a visualizare la sezione dedicata agli autori
\end{itemize} \\
  \hline
\end{tabular}\\

\subsubsection{Scenario CreaConferenza}
% DONE: Rifare scenario Crea Conferenza
\begin{tabular}{|p{3cm}|p{7cm}|}
\hline 
\rowcolor{Orchid}
Caso d'uso & Crea Conferenza \\
\hline
Attore Primario & Organizzatore\\
\hline
Attore Secondario & \\
\hline
Descrizione & L'organizzatore crea una conferenza \\
\hline
  Sequenza Eventi &
                    \begin{enumerate}
                      \item L'organizzatore inserisce Titolo, Descrizione e la data di scadenza per la sottomissione degli articoli
                      \item Il sistema verifica che la data inserita sia valida
                      \item IF la data inserita è valida
                      \begin{enumerate}
                        \item Il sistema controlla che la data inserita non sia nel passato
                        \item IF la data inserita non è nel passato
                        \begin{enumerate}
                          \item Il sistema salva la conferenza
                          \item Il sistema rende la conferenza disponibile per la sottomissione di articoli
                        \end{enumerate}
                        \item ALTRIMENTI Il sistema mostra un messaggio di errore
                      \end{enumerate}
                    \item ALTRMIENTI Il sistema mostra un messaggio di errore
                    \end{enumerate}\\
\hline
Post Condizioni: & L'organizzatore crea una conferenza a cui i vari autori possono sottomettere degli articoli \\
\hline
Sequenza degli eventi alternativa &
                                    \begin{itemize}
                                      \item Nel punto 3 se la data inserita è in un formato non valido viene mostrato un messaggio di errore con il formato da utilizzare e viene chiesta una nuova scadenza
                                      \item Nel punto 3.2 se la data inserita è nel passato viene mostrato un messaggio di errore e viene chiesta una nuova scadenza
                                      \end{itemize}\\
\hline
\end{tabular}\\


\subsubsection{Scenario SottomettiArticolo}
% TODO: Rifare scenario SottomettiArticolo
\begin{tabular}{|p{3cm}|p{7cm}|}
\hline 
\rowcolor{Orchid}
Caso d'uso & SottomettiArticolo \\
\hline
  Attore Primario & Autore\\
  \hline
  Attore Secondario & \\
\hline
Descrizione & L'autore sottomette un articolo per una determinata conferenza\\
\hline
Pre-Condizioni& Esiste la conferenza\\
\hline
  Sequenza Eventi &
                    \begin{enumerate}
                    \item L'autore inserisce un articolo composto da un Titolo,Abstract e dai Co-Autori
                    \item Il sistema inserisce l'articolo e verifica che l'autore e i co-autori siano registrati al sistema
                    \item Se il controllo è positivo e non si è verifcato alcun errrore
                      \begin{enumerate}
                      \item Il sistema informa all'utente la buona uscita dell'operazione
                      \item Il sistema inserisce l'articolo e permette agli autori di poter visualizzare lo stato dell'articolo
                      \end{enumerate}
                    \end{enumerate} \\
\hline
Post-Condizioni: &L'articolo è inserito e possibile per la verica degli articoli da parte dei Revisori\\
\hline
Sequenza degli eventi alternativa & Il sistema nel caso in cui i co-autori non sono stati registrati non procede con l'inserimento dell'articolo e restituisce un messaggio di errore\\
\hline
\end{tabular}


\begin{tabular}{|p{3cm}|p{7cm}|}
\hline 
\rowcolor{Orchid}
Caso d'uso & AssegnaRevisori \\
\hline
Attore Primario & Organizzatore\\
\hline
Attore Secondario & Sistema di notifica\\
\hline
Descrizione &L'organizzatore  assegna ad ogni articolo dei revisori per la revisione dell'articolo\\
\hline
Pre-Condizioni& L'Autore ha sottomesso l'articolo\\
\hline
  Sequenza Eventi&
                   \begin{enumerate}
                   \item L'organizzatore seleziona l'articolo da revisionare
                   \item L'organizzatore seleziona la data di scadanza della revisione
                   \item L'organizzatore va selezionare i revisore
                   \item IF vine rispetata la data di scadenza
                   \begin{enumerate}
                    \item IF VerificaConflittiInterrese va buon fine
                    \begin{enumerate}
                      \item Il sitema aggiorna lo stato dell'articolo passa da "sottomesso" a "in revisione"
                      \item I revisori selezionati vengono assegnati all'articolo
                      \item IF InserimentoEsitoRevisione va buon fine
                      \begin{enumerate}
                        \item Il sitema aggiorna lo stato dell'articolo da "in revisione" ad "approvato"
                      \end{enumerate}
                      \item ALTRIMENTI lo stato dell'articolo passa da "in revisione" ad "rifiutato"
                    \end{enumerate}
                    \item ALTRIMENTI l'organizzatore deve selezionare altri revisori
                   \end{enumerate}
                   \item ALTRIMENTI lo stato dell'articolo passa da "in revisione" ad "rifiutato"
                   \end{enumerate}\\
\hline
Post-Condizioni: &L'articolo passa da uno stato di "sottomesso" a quello di "revisione" \\
\hline
Sequenza degli eventi alternativa & \begin{itemize}
\item Nel punto 4 se la data di scadenza della revisione non viene rispetata, l'articolo viene automaticamente rifiutato
\item Nel punto 4.1 se la VerificaConflittiInterrese risulta negativo, l'organizzatore deve selezionare altri revisori
\item Nel punto 4.1.3 se InserimentoEsitoRevisione ha esito negativo, l'articolo automaticamente rifiutato
\end{itemize}\\
\hline
\end{tabular}\\

% TODO: Visionare scenario InserimentoEsitoRevisione
\begin{tabular}{|p{3cm}|p{7cm}|}
\hline 
\rowcolor{Orchid}
Caso d'uso & Inserimento Esito Revisione\\
\hline
Attore Primario & Autore \\
\hline
Attore Secondario & \\
\hline
Descrizione & L'autore selezionato come revisore inserisce esito della revisione\\
\hline
Pre-Condizioni& L'articolo è in fase di revisione\\
\hline
  Sequenza Eventi&
                   \begin{enumerate}
                   \item Il revisore inserisce esito della revisore e descrizione
                   \item IF l'esito di tutti i revisori dell'articolo è positivo
                   \begin{enumerate}
                    \item Il sistema aggiorna lo stato dell'articolo da "in revisione" ad "approvato" 
                    \item L'articolo viene viene inserito alla conferenza
                   \end{enumerate}
                   \item ALTRIMENTI Il sistema aggiorna lo stato dell'articolo da "in revisione" ad "rifiutato"
                   \item Il sistema notifica l'autore dell'esito della revisione
                   \end{enumerate}\\
\hline
Post-Condizioni: & L'articolo viene aggiornato con l'esito della revisione\\
\hline
Sequenza degli eventi alternativa & \begin{itemize}
  \item Nel punto 2 se uno dei revisori da come esito della revisione negativo l'articolo viene rifiutato
\end{itemize} \\
\hline
\end{tabular}

% TODO: Scrivere scenario VerificaConflittiInterrese

\begin{tabular}{|p{3cm}|p{7cm}|}
\hline 
\rowcolor{Orchid}
Caso d'uso & NotificaScadenza \\
\hline
Attore Primario & VerificaConflittiInterrese\\
\hline
Attore Secondario & \\
\hline
Descrizione & \\
\hline
Pre-Condizioni& \\
\hline
  Sequenza Eventi&
                  \begin{enumerate}
                   \item
                  \end{enumerate}\\
\hline
Post-Condizioni: & 
\hline
Sequenza degli eventi alternativa & \begin{itemize}
  \item 
\end{itemize} \\
\hline
\end{tabular}

\subsubsection{Scenario NotificaScadenza}
\begin{tabular}{|p{3cm}|p{7cm}|}
\hline 
\rowcolor{Orchid}
Caso d'uso & NotificaScadenza \\
\hline
Attore Primario & Sistema di Notifica\\
\hline
Attore Secondario & \\
\hline
Descrizione & Permette l'invio automatico di email a tutti gli autori per conferenze la scadenza è imminente\\
\hline
Pre-Condizioni& Che una conferenza risulta in scadenza\\
\hline
  Sequenza Eventi&
                   \begin{enumerate}
                   \item Il sistema verifica l'imminenzza della scadenza di una conferenzza
                   \item IF la conferenza è in scadenza
                   \item Il sistema invia delle notifiche che la conferenza è in scadenza
                   \item Ogni autore riceve una notifica che l'informa che la conferenza è in scadenza
                   \item Altrimenti non viene inviata nessuana notifica
                   \end{enumerate}\\
\hline
Post-Condizioni: & Il sistema invia automaticamente l'email per la notifica della scadenzza di una conferenzza\\
\hline
Sequenza degli eventi alternativa & \begin{itemize}
  \item Nel punto 2 se non c'è nessuna conferenza in scadenza non viene inviata nessuan email
\end{itemize} \\
\hline
\end{tabular}

% TODO: Scrivere Scenario InserimentoEsitoRevisione
\begin{tabular}{|p{3cm}|p{7cm}|}
\hline 
\rowcolor{Orchid}
Caso d'uso & Inserimento Esito Revisione\\
\hline
Attore Primario & Autore \\
\hline
Attore Secondario & \\
\hline
Descrizione & L'autore selezionato come revisore inserisce esito della revisione\\
\hline
Pre-Condizioni& \\
\hline
  Sequenza Eventi&
                   \begin{enumerate}
                   \item Il revisore inserisce esito della revisore e descrizione
                   \item IF l'esito di tutti i revisori dell'articolo è positivo
                   \item Il sistema aggiorna lo stato dell'articolo
                   \item L'articolo viene accettato
                   \item ALTRIMENTI l'articolo viene rifiutato
                   \item Il sistema notifica l'autore dell'esito della revisione
                   \end{enumerate}\\
\hline
Post-Condizioni: & \\
\hline
Sequenza degli eventi alternativa & \begin{itemize}
  \item Nel punto 2 se uno devi revisore da come esito della revisione negativo l'articolo viene rifiutato
\end{itemize} \\
\hline
\end{tabular}
% TODO: Visionare scenario InserimentoEsitoRevisione
\begin{tabular}{|p{3cm}|p{7cm}|}
\hline 
\rowcolor{Orchid}
Caso d'uso & Inserimento Esito Revisione\\
\hline
Attore Primario & Autore \\
\hline
Attore Secondario & \\
\hline
Descrizione & L'autore selezionato come revisore inserisce esito della revisione\\
\hline
Pre-Condizioni& L'articolo è in fase di revisione\\
\hline
  Sequenza Eventi&
                   \begin{enumerate}
                   \item Il revisore inserisce esito della revisore e descrizione
                   \item IF l'esito di tutti i revisori dell'articolo è positivo
                   \item Il sistema aggiorna lo stato dell'articolo
                   \item L'articolo viene accettato
                   \item ALTRIMENTI l'articolo viene rifiutato
                   \item Il sistema notifica l'autore dell'esito della revisione
                   \end{enumerate}\\
\hline
Post-Condizioni: & L'articolo viene aggiornato con l'esito della revisione\\
\hline
Sequenza degli eventi alternativa & \begin{itemize}
  \item Nel punto 2 se uno dei revisori da come esito della revisione negativo l'articolo viene rifiutato
\end{itemize} \\
\hline
\end{tabular}

%%% Local Variables:
%%% mode: LaTeX
%%% TeX-master: "main"
%%% End:
