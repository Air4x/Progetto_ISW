\subsection{Scenari}
\label{sec:scenari}
Per semplicita abbiamo deciso di affrontare i seguenti casi d'uso:
\begin{itemize}
  \item Accesso
\item Registrazione
\item CreaConferenza
\item NotificaScadenze
\item AssengaRevisori
\end{itemize}
% DONE: Rifare i scenari


\subsubsection{Scenario Registrazione}
\label{sec:scenari:registrazione}
% DONE: Rifare scenario Registrazione
\begin{tabular}{|p{3cm}|p{7cm}|}
\hline 
\rowcolor{Orchid}
  Caso d'uso & Registrazione \\
  \hline
  Attore Primario & Utente\\
  \hline
Attore Secondario & \\
\hline
Descrizione & Permette la registrazione di un utente\\
\hline
  Sequenza Eventi &
    \begin{enumerate}
    \item Lo scenario inizia quando l'utente clicca sul bottone Registrarsi
    \item L'utente inserisce Nome, Cognome, Email, Password, affiliazione
    \item L'utente sceglie il proprio ruolo
    \item L'utente clicca sul bottone registrarsi
    \item Il sistema verifica la validità dell'indirizzo email
    \item IF l'indirizzo email è valido
   \begin{enumerate}
    \item il sistema controlla che l'indirizzo non sia già presente
     \item IF indirizzo email non è presente
       \begin{enumerate}
         \item il sistema controlla la validità della password
           \begin{enumerate}
            \item IF la password è valida
               \begin{enumerate}
               \item Il sistema salva i dati dell'Utente
               \item punto di estensione: GestionePermessiVisibilià
               \end{enumerate}
            \item ALTRIMENTI il Sistema mostra un messaggio di errore
          \end{enumerate}
        \item ALTRIMENTI il Sistema mostra un messaggio di errore spiegando che non è possibile usare l'email
       \end{enumerate}         
      \end{enumerate}
 \item ALTRIMENTI il Sistema mostra un messaggio di errore spiegando che l'email non è valida
  \end{enumerate}\\
\hline
Post-Condizioni: & Il cliente è registrato nel sistema \\
  \hline
\end{tabular}\\ 
\begin{tabular}{|p{3cm}|p{7cm}|}
  \hline
  Sequenza degli eventi alternativa & \begin{itemize}
  \item Se nel punto 5 l'indirizzo email non è valido viene mostrato un messaggio di errore all'utente e gli si chiede di inserire un nuovo indirizzo email
\item Se nel punto 5.2 l'indirizzo email è già presente nel sistema si mostra un messaggio di errore e si chiede all'utente di usare credenziali diverse
\item Se nel punto 5.2.1.1 la password risulta non valida viene mostrato un messaggio di errore all'utente e gli viene chiesto di inserire una nuova password
  \end{itemize}\\
\hline
\end{tabular} \\

\subsubsection{Scenario Accesso}
% DONE: Rifare scenario Accesso
\begin{tabular}{|p{3cm}|p{7cm}|}
\hline 
\rowcolor{Orchid}
Caso d'uso & Accesso \\
\hline
Attore Primario & Utente\\
\hline
Attore Secondario & \\
\hline
Descrizione & Permette l'accesso di un utente al suo account\\
\hline
Pre-Condizioni& L'utente sia già registrato\\
\hline
  Sequenza Eventi&
\begin{enumerate}
\item Lo scenario inizia quando l'utente clicca sul bottone Accedi
\item L'utente inserisce email e password
\item Il sistema controlla la validità dell'email
\item IF l'email inserita è valida
\begin{enumerate}
  \item Il sistema controlla se l'email è presente nel sistema
  \item IF l'email è presente nel sistema
\begin{enumerate}
  \item il sistema controlla se le credenziali sono corrette
\item IF le credenziali sono corrette
\begin{enumerate}
\item punto di estensione: GestionePermessiVisibilità
\item IF la GestionePermessiVisibilità va a buon fine
\begin{enumerate}
  \item mostra all'utente la giusta schermata per il suo ruolo
\end{enumerate}
\end{enumerate}
\end{enumerate}
\end{enumerate}
\end{enumerate}\\
\hline
Post-Condizioni: &il cliente ha l'accesso al suo account \\
  \hline
\end{tabular}\\
\begin{tabular}{|p{3cm}|p{7cm}|}
\hline
  Sequenza degli eventi alternativa & \begin{itemize}
  \item Nel punto 3 se l'email inserita non è valida viene mostrato un messaggio di
errore e viene chiesto all'utente di inserire un nuovo indirizzo email
\item Nel punto 3.2 se l'indirizzo email non è presente nel sistema viene mostrato un
messaggio di errore e viene chiesto all'utente di inserire delle credenziali corrette
\item Nel punto 3.2.2 se le credenziali non sono corrette viene mostrato un messaggio
di errore e viene chiesto all'utente di inserire delle credenziali corrette
\item Nel punto 3.2.2.2 se la GestioneVisibilitàPermessi non va a buon fine
viene mostrato un messaggio di errore
\end{itemize}\\
\hline
\end{tabular}\\

\subsubsection{Scenario CreaConferenza}
% DONE: Rifare scenario Crea Conferenza
\begin{tabular}{|p{3cm}|p{7cm}|}
\hline 
\rowcolor{Orchid}
Caso d'uso & Crea Conferenza \\
\hline
Attore Primario & Organizzatore\\
\hline
Attore Secondario & \\
\hline
Descrizione & L'organizzatore crea una conferenza \\
\hline
  Sequenza Eventi &
                    \begin{enumerate}
                      \item Lo scenario organizzatore clicca il tasto crea conferenza
                      \item L'organizzatore inserisce Titolo, Descrizione e la data di scadenza per la sottomissione degli articoli
                      \item Il sistema verifica che la data inserita sia valida
                      \item IF la data inserita è valida
                      \begin{enumerate}
                        \item Il sistema controlla che la data inserita non sia nel passato
                        \item IF la data inserita non è nel passato
                        \begin{enumerate}
                          \item Il sistema salva la conferenza
                          \item Il sistema rende la conferenza disponibile per la sottomissione di articoli
                        \end{enumerate}
                      \end{enumerate}
                    \end{enumerate}\\
\hline
Post Condizioni: & L'organizzatore crea una conferenza a cui i vari autori possono sottomettere degli articoli \\
\hline
Sequenza degli eventi alternativa &
                                    \begin{itemize}
                                      \item Nel punto 3 se la data inserita è in un formato non valido viene mostrato un messaggio di errore con il formato da utilizzare e viene chiesta una nuova scadenza
                                      \item Nel punto 3.2 se la data inserita è nel passato viene mostrato un messaggio di errore e viene chiesta una nuova scadenza
                                      \end{itemize}\\
\hline
\end{tabular}\\
% DONE: Rifare scenario SottomettiArticolo
\subsubsection{Scenario AssegnaRevisori}
\begin{tabular}{|p{3cm}|p{7cm}|}
\hline 
\rowcolor{Orchid}
Caso d'uso & AssegnaRevisori\\
\hline
Attore Primario & Organizzatore\\
\hline
Attore Secondario & \\
\hline
Descrizione &L'organizzatore  assegna ad ogni articolo dei revisori per la revisione dell'articolo\\
\hline
Pre-Condizioni& L'Autore ha sottomesso l'articolo\\
\hline
  Sequenza Eventi&
                   \begin{enumerate}
                     \item Lo scenario inizia quando l'organizzatore seleziona l'articolo da revisionare
\item L'organizzatore clicca sul tasto assegna revisore
                   \item L'organizzatore seleziona la data di scadanza della revisione
                   \item L'organizzatore va selezionare i revisore
                   \item IF vine rispetata la data di scadenza
                   \begin{enumerate}
                    \item IF VerificaConflittiInterrese va buon fine
                    \begin{enumerate}
                      \item Il sitema aggiorna lo stato dell'articolo passa da "sottomesso" a "in revisione"
                      \item IF InserimentoEsitoRevisione va buon fine
                      \begin{enumerate}
                        \item Il sitema aggiorna lo stato dell'articolo da "in revisione" ad "approvato"
                      \end{enumerate}
                      \item ALTRIMENTI lo stato dell'articolo passa da "in revisione" ad "rifiutato"
                    \end{enumerate}
                    \item ALTRIMENTI l'organizzatore deve selezionare altri revisori
                   \end{enumerate}
                   \item ALTRIMENTI lo stato dell'articolo passa da "in revisione" ad "rifiutato"
                   \end{enumerate}\\
\hline
Post-Condizioni: &L'articolo passa da uno stato di "sottomesso" a quello di "revisione" \\
\hline
Sequenza degli eventi alternativa & \begin{itemize}
\item Nel punto 4 se la data di scadenza della revisione non viene rispetata, l'articolo viene automaticamente rifiutato
\item Nel punto 4.1 se la VerificaConflittiInterrese risulta negativo, l'organizzatore deve selezionare altri revisori
\item Nel punto 4.1.2 se InserimentoEsitoRevisione ha esito negativo, l'articolo automaticamente rifiutato
\end{itemize}\\
\hline
\end{tabular}

% DONE: Rifare scenario InserimentoEsitoRevisione
\subsubsection{Scenario VerificaConflittiInterrese}
\begin{tabular}{|p{3cm}|p{7cm}|}
\hline 
\rowcolor{Orchid}
Caso d'uso & VerificaConflittiInterrese \\
\hline
Attore Primario & \\
\hline
Attore Secondario & \\
\hline
Descrizione & Il sistema va a verificare che non ci siano conflitti tra autori che scrivono l'articolo e autori che vanno a revisionare l'articolo\\
\hline
Pre-Condizioni& L'articolo risulti sottomesso a una conferenza\\
\hline
  Sequenza Eventi&
                  \begin{enumerate}
                   \item Il sistema va verificare che autori dell'articolo non coicidono con gli revisore
                   \item IF verifica va buon fine
                    \begin{enumerate}
                      \item Il sitema va ad assegnare l'articolo agli revisori
                    \end{enumerate}
                    \item ALTRIMENTI il sistema va a mostrare un messaggio di errore
                  \end{enumerate}\\
\hline
Post-Condizioni: & L'articolo è stato assegnato ai revisori\\
\hline
Sequenza degli eventi alternativa & \begin{itemize}
  \item Nel punto 2 se la verifica non va buon fine l'articolo non viene assegnato ai revisori
\end{itemize} \\
\hline
\end{tabular}

% DONE: Rifare scenario NotificaScadenza
\subsubsection{Scenario NotificaScadenza}
\begin{tabular}{|p{3cm}|p{7cm}|}
\hline 
\rowcolor{Orchid}
Caso d'uso & NotificaScadenza \\
\hline
Attore Primario & Sistema di notifica\\
\hline
Attore Secondario & \\
\hline
Descrizione & Permette l'invio automatico di email a tutti gli autori per conferenze la scadenza è imminente\\
\hline
Pre-Condizioni& Che una conferenza risulta in scadenza\\
\hline
  Sequenza Eventi&
                   \begin{enumerate}
                     \item Lo scenario inizia allo scadere di un timer interno del sistema una volta al giorno
                   \item Il sistema verifica l'imminenzza della scadenza di una conferenzza
                   \item IF la conferenza è in scadenza
                    \begin{enumerate}
                    \item Il sistema invia delle notifiche che la conferenza è in scadenza
                    \item Ogni autore riceve una notifica che l'informa che la conferenza è in scadenza
                    \end{enumerate}
                   \item Altrimenti non viene inviata nessuana notifica
                   \end{enumerate}\\
\hline
Post-Condizioni: & Il sistema invia automaticamente l'email per la notifica della scadenzza di una conferenzza\\
\hline
Sequenza degli eventi alternativa & \begin{itemize}
  \item Nel punto 2 se non c'è nessuna conferenza in scadenza non viene inviata nessuan email
\end{itemize} \\
\hline
\end{tabular}

%%% Local Variables:
%%% mode: LaTeX
%%% TeX-master: "main"
%%% End:
