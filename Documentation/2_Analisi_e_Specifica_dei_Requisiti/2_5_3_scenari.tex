\subsection{Scenari}
% TODO: Rifare i scenari
\label{sec:scenari}

\subsubsection{Scenario Registrazione}
% DONE: Rifare scenario Registrazione
\begin{tabular}{|p{3cm}|p{7cm}|}
\hline 
\rowcolor{Orchid}
Caso d'uso & Registrazione \\
\hline
Attore Primario & Utente\\
\hline
Attore Secondario & \\
\hline
Descrizione & Permette la registrazione di un utente\\
\hline
  Sequenza Eventi &
  \begin{enumerate}
    \item L'utente inserisce Nome, Cognome, Email, Password, affiliazione
    \item L'utente sceglie il proprio ruolo
    \item L'utente clicca sul bottone registrarsi
    \item Il sistema verifica la validità dell'indirizzo email
    \item IF l'indirizzo email è valido
   \begin{enumerate}
    \item il sistema controlla che l'indirizzo non sia già presente
     \item IF indirizzo email non è presente
       \begin{enumerate}
         \item il sistema controlla la validità della password
           \begin{enumerate}
            \item IF la password è valida
               \begin{enumerate}
               \item Il sistema salva i dati dell'Utente
               \item punto di estensione: GestionePermessiVisibilià
               \end{enumerate}
            \item ALTRIMENTI il Sistema mostra un messaggio di errore
          \end{enumerate}
        \item ALTRIMENTI il Sistema mostra un messaggio di errore spiegando che non è possibile usare l'email
       \end{enumerate}         
      \end{enumerate}
 \item ALTRIMENTI il Sistema mostra un messaggio di errore spiegando che l'email non è valida
  \end{enumerate}\\
\hline
Post-Condizioni: & Il cliente è registrato nel sistema \\
  \hline
\end{tabular}\\ 
\begin{tabular}{|p{3cm}|p{7cm}|}
  \hline
  Sequenza degli eventi alternativa & \begin{itemize}
  \item Se nel punto 5 l'indirizzo email non è valido viene mostrato un messaggio di errore all'utente e gli si chiede di inserire un nuovo indirizzo email
\item Se nel punto 5.2 l'indirizzo email è già presente nel sistema si mostra un messaggio di errore e si chiede all'utente di usare credenziali diverse
\item Se nel punto 5.2.1.1 la password risulta non valida viene mostrato un messaggio di errore all'utente e gli viene chiesto di inserire una nuova password
  \end{itemize}\\
\hline
\end{tabular} \\

\subsubsection{Scenario Accesso}
% TODO: Rifare scenario Accesso
\begin{tabular}{|p{3cm}|p{7cm}|}
\hline 
\rowcolor{Orchid}
Caso d'uso & Accesso \\
\hline
Attore Primario & Utente\\
\hline
Attore Secondario & \\
\hline
Descrizione & Permette l'accesso ad un utente al suo account\\
\hline
Pre-Condizioni& L'utente sia già registrato\\
\hline
  Sequenza Eventi&
                   \begin{enumerate}
                   \item Il caso d'uso inizia quando l'Utente inserisce Email E Password
                   \item Il sistema verifica che l'email e  passoword coincidono con l'utente già registrato
                   \item Se il controllo è positivo e non si è verifcato alcun errrore
                     \begin{enumerate}
                     \item il sistema informa all'utente la buona uscita dell'operazione
                     \end{enumerate}
                   \end{enumerate}\\
\hline
Post-Condizioni: &il cliente ha l'accesso al suo account \\
\hline
Sequenza degli eventi alternativa & Al punto 3 il cliente immette una Email collegato ad alcun account  oppure che l'email e passoword non coincidono restituendo un messaggio di errore\\
\hline
\end{tabular}

\begin{tabular}{|p{3cm}|p{7cm}|}
\hline 
\rowcolor{Orchid}
Caso d'uso & Crea Conferenza \\
\hline
Attore Primario & Organizzatore\\
\hline
Attore Secondario & Sistema Di notifica\\
\hline
Descrizione & L'organizzatore crea una conferenza \\
\hline
  Sequenza Eventi &
                    \begin{enumerate}
                    \item L'organizzatore crea una Conferenza inserendo il Titolo, una Descrizione dell'evento e una singola data di scadenza per le sottomissioni
                    \item Il sistema verifica che la data non sia passata o non sia valida
                    \item Se il controllo è positivo e non si è verifcato alcun errrore
                      \begin{enumerate}
                      \item il sistema informa all'utente la buona uscita dell'operazione
                      \item Il sistema permette il monitoraggio dei vari articoli sottomessi per la conferenza
                      \end{enumerate}
                    \end{enumerate}\\
\hline
Post Condizioni: & L'organizzatore crea una conferenza a cui i vari autori possono sottomettere degli articoli \\
\hline
Sequenza degli eventi alternativa & Nel caso in cui la data di scadenza inserita non sia valida o che sia già passata invia un messaggio di errore \\
\hline
\end{tabular}

\begin{tabular}{|p{3cm}|p{7cm}|}
\hline 
\rowcolor{Orchid}
Caso d'uso & SottomettiArticolo \\
\hline
Attore Primario & Autore\\
\hline
Descrizione & L'autore sottomette un articolo per una determinata conferenza\\
\hline
Pre-Condizioni& Esiste la conferenza\\
\hline
  Sequenza Eventi &
                    \begin{enumerate}
                    \item L'autore inserisce un articolo composto da un Titolo,Abstract e dai Co-Autori
                    \item Il sistema inserisce l'articolo e verifica che l'autore e i co-autori siano registrati al sistema
                    \item Se il controllo è positivo e non si è verifcato alcun errrore
                      \begin{enumerate}
                      \item Il sistema informa all'utente la buona uscita dell'operazione
                      \item Il sistema inserisce l'articolo e permette agli autori di poter visualizzare lo stato dell'articolo
                      \end{enumerate}
                    \end{enumerate} \\
\hline
Post-Condizioni: &L'articolo è inserito e possibile per la verica degli articoli da parte dei Revisori\\
\hline
Sequenza degli eventi alternativa & Il sistema nel caso in cui i co-autori non sono stati registrati non procede con l'inserimento dell'articolo e restituisce un messaggio di errore\\
\hline
\end{tabular}

\begin{tabular}{|p{3cm}|p{7cm}|}
\hline 
\rowcolor{Orchid}
Caso d'uso & AssegnaRevisori \\
\hline
Attore Primario & Organizzatore\\
\hline
Attore Secondario & Sistema di notifica\\
\hline
Descrizione &L'organizzatore  assegna ad ogni articolo dei revisori per la revisione dell'articolo\\
\hline
Pre-Condizioni& L'Autore ha sottomesso l'articolo\\
\hline
  Sequenza Eventi&
                   \begin{enumerate}
                   \item Il caso d'uso inizia allo scadere della data di revisione
                   \item L'organizzatore assegna ad ogni articolo sottomesso tre revisori
                   \item L'assegnazione può essere tramite ID oppure scegliendoli dalla lista di quelli registrati
                   \end{enumerate}\\
\hline
Post-Condizioni: &L'articolo passa da uno stato di "sottomesso" a quello di "revisione" \\
\hline
Sequenza degli eventi alternativa & \\
\hline
\end{tabular}
  

\begin{tabular}{|p{3cm}|p{7cm}|}
\hline 
\rowcolor{Orchid}
Caso d'uso & NotificaScadenza \\
\hline
Attore Primario & Sistema di Notifica\\
\hline
Attore Secondario & Autore\\
\hline
Descrizione & Permette la notifica automatica via email per le principali scadenze\\
\hline
Pre-Condizioni& L'utente sia già registrato\\
\hline
  Sequenza Eventi&
                   \begin{enumerate}
                   \item Il caso d'uso inizia quando la scadenza è imminente
                   \item Il sistema invia delle notifiche agli autori per le scadenze delle varie conferenze
                   \item Tramite Email vengono notificati gli autori
                   \end{enumerate}\\
\hline
Post-Condizioni: & \\
\hline
Sequenza degli eventi alternativa & \\
\hline
\end{tabular}

%%% Local Variables:
%%% mode: LaTeX
%%% TeX-master: "main"
%%% End:
