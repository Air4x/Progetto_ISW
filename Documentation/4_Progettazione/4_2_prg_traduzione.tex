\subsection{Traduzione classi ed associazioni}
\label{sec:prg_traduzione}

In tale \textbf{sezione} si va esplicitare il processo che dal modello concettuale ci porta al modello di progettazione, basato sull'architettura \textbf{BCED} (\textbf{Boundary}, \textbf{Controller}, \textbf{Entity}, \textbf{Database}).
Dove le scelte di progettazione sono state influenzzate da:
\begin{itemize}
\item Vincoli e le funzionalità espresse nei requisiti \ref{sec:analisi_nomi_verbi}
\item Rispettare i vicoli del modello \textbf{BCED}
\end{itemize}

\subsubsection{Classi Entity}
\begin{description}
\item [User] Generalizzazione del concetto di utente, classe astratta, poiche nel sistema un user può essere solo o un \textbf{author} o un \textbf{organizer}. Caratterizatto dagli segueni attributi:
\begin{itemize}
\item ID: Che va a indetificare in maniera \textbf{univoca} un user
\item Name: nome dell'user
\item Lastname: cognome dell'user
\item Email: elemento identificativo per l'accesso al database per l'user
\item Password:  elemento identificativo per l'accesso al database per l'user 
\item Affiliation: organnizazione di appartenenza dell'user
\end{itemize}
\item [Author] Estende la classe \textbf{User}, ereditando tutti gli attributi e metodi. Caratterizzato dall'attributo:
\begin{itemize}
\item Final-Role: indica il ruolo dell'user (in questo caso \textbf{Autore})
\end{itemize}
\item [Organizer] Estende la classe \textbf{User}, ereditando tutti gli attributi e metodi. Caratterizzato dall'attributo:
\begin{itemize}
\item Final-Role: indica il ruolo dell'user (in questo caso \textbf{Organizzatore}) 
\end{itemize}
\item [Conference] rappresenta una conferenza creata da un \textbf{organizer}, caratterizzata dagli seguenti attributi:
\begin{itemize}
\item ID: Che va a indetificare in maniera \textbf{univoca} un conferenza
\item Title: Il nome della conferenzza
\item Description: descrizione dell'argomento su cui è incetratta la conferenzza
\item Deadline: data di scadenzza per la consegna degli articoli
\item Organizer: Che va a indetificare in maniera \textbf{univoca} l'organnizatore della conferenza
\item Articles: lista di riferimenti agli articoli sottomessi
\end{itemize}
\item [Article] rappresenta un articolo sottomesso, caratterizzato dagli seguenti attributi:
\begin{itemize}
\item ID: Che va a indetificare in maniera \textbf{univoca} un articolo
\item Title: Titolo dell'articolo
\item Abstr: Contenuto dell'articolo
\item Authors: lista di riferimenti degli auto che hanno partecipato alla creazione dell'articolo
\end{itemize}
\end{description}

\paragraph{Associazioni}
\begin{description}
\item [Organizer->Conference] associazione di tipo: \textbf{1->* [uno a molti]} indica che uno stesso organizzatore può creare più conferenze
\item [Conference->Article] associazione di tipo: \textbf{1->* [uno a molti]} dove una confderenza può avere da uno a più articoli
\item [Author->Article] associazione di tipo: \textbf{*->* [molti a molti]} dove un articolo può essere scrito da uno o più autori mentre quest'ultimi possono scrivere più articoli.
\item [Article->Author] associazione di tipo: \textbf{*->3 [molti a molti]} con cui ad un aticolo si possono assegnare un messimo di 3 revisore attraverso ReviewDAO e ReviewController in particolar modo con:
\begin{itemize}
 \item assignReviewer(articleId, reviewerId)
\item getReviewersForArticle(articleId)
\end{itemize}
\end{description}

% TODOPROF: Far visionare alla prof la traduzioone del package controller e dto.
% DONE: Scrivere la traduzione per classi e associazioni per il package Controller
\subsubsection{Classi Controller}
Le classi del package controller vanno a gestire la logica di business occupando il ruolo di tramite tra le classi del package Entity e le classi del package Boundary.
\begin{description}
\item[UserController] Classe che implementa la logica di gestione utenti.
    \begin{itemize}
        \item registerUser (...): Metodo per la creazione di un nuovo utente.
        \item login (...): Metodo per l'accesso di un utente.
        \item getRAuthorBYEmail (...): Metodo per ottenere un autore mediante email.
        \item getCooAuthors(...): Metodo per ottenere un autore mediante email.
    \end{itemize}
\item[ConferenceController] Classe che implementa la logica di gestione delle Conferenzze.
    \begin{itemize}
        \item createConference (...): Metodo per la creazione di una conferenzza.
        \item getActiveConferences(): Metodo per ottenere una lista di conferenzze attive [NON ANCORA SCADUTE].
        \item getArticlesByConference(...): Metodo per ottenere una lista di articoli.
    \end{itemize}
\item[ArticleController] Classe che implementa la logica di gestione degli Articoli.
    \begin{itemize}
        \item submitArticle(...): Metodo per la sottomisione di un articolo a una conferenza.
        \item getArticleByAuthor(...): Metodo per ottenere un articolo attraverso gli autori.
    \end{itemize}
\item[NotificationController] Classe che estende java.util.TimerTask Classe che implementa la logica di gestione della creazione e invio di email
    \begin{itemize}
        \item run(): Override del metodo run della classe TimerTask per esecuzione ad intervalli regolari per l'invio delle notifiche.
        \item sendNotificationDeadline(): Metodo utilizzato per l'invio di email.
        \item createMessageExpireConference(...): Metodo per la creazione del messaggio da inviare.
        \item sendEmail (...): Metodo utilizzo per la configurazione di un host per l'invio delle email.
    \end{itemize}
\end{description}

% DONE: Scrivere la traduzione per classi e associazioni per il package DTO
\subsubsection{Classi Data Transfer Object}
\begin{description}
\item[RUserDTO] Classe DTO per il trasporto delle informazioni di un utente
\begin{itemize}
\item ID: Che va a indetificare in maniera \textbf{univoca} un user
\item Name: nome dell'user
\item Lastname: cognome dell'user
\item Email: elemento identificativo per l'accesso al database per l'user
\item Ruolo: che rapresenta il ruolo occupato dall'user
\item Affiliazzione: organnizazione di appartenenza dell'user
\end{itemize}
\item[PossibleReviewDTO] Classe DTO per il trasporto delle informazioni di un revisore [autore]
\begin{itemize}
\item ID: Che va a indetificare in maniera \textbf{univoca} un user
\item Name: nome dell'user
\item Lastname: cognome dell'user
\item Affiliazzione: organnizazione di appartenenza dell'user
\end{itemize}
\item[ShowArticleDTO] Classe DTO per il trasporto delle informazioni di un articolo
\begin{itemize}
\item ID: Che va a indetificare in maniera \textbf{univoca} un articolo
\item Titolo: Titolo dell'articolo
\item Abstr: Contenuto dell'articolo
\item Autori: lista di riferimenti degli auto che hanno partecipato alla creazione dell'articolo
\end{itemize}
\item[ShowActiveConferenceDTO]
\begin{itemize}
\item ID: Che va a indetificare in maniera \textbf{univoca} un conferenza
\item Title: Il nome della conferenzza
\item Description: descrizione dell'argomento su cui è incetratta la conferenzza
\item Deadline: data di scadenzza per la consegna degli articoli
\end{itemize}
\end{description}

La notazione [...] è stata utilizzata per motivi di spazio per vedere la dati di ingresso dei vari metodi si prega di andare a vedere:
\begin{itemize}
    \item Per le classi del package controller \ref{sec:package_controller_prg_image}
\end{itemize}

%%% Local Variables:
%%% mode: LaTeX
%%% TeX-master: "main"
%%% End:
