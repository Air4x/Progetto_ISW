\subsubsection{Package Boundary}

\label{fig:package_boundary_prg_image}

\begin{figure}[ht]
  \centering
  \includegraphics[width=\linewidth]{VisualParadigm/diagramma_prg_boundary.png}
  \caption{Package Boundary}
  \label{fig:Package Boundary}
\end{figure}

Nel Package Boundary vi sono tutti gli oggetti responsabili della costruzione della GUI e della logica di presentazione. A questo livello tutte le classi corrispondono a delle interfacce. 

\begin{description}
\item[AssignReviewsView] :Classe che implementa l'interfaccia grafica per permettere all'organizzatore di assegnare i revisori
\item[CreateConferenceForm] :Classe che implementa l'interfaccia grafica per permettere la creazione di una nuova Conferenza
\item[LoginView] :Classe che implementa l'interfaccia grafica che implementa la funzione per l'accesso
\item[RegistrationForm] :Classe che implementa l'interfaccia grafica per la registrazione
\item[AuthorDashboard] :Classe che implementa l'interfaccia grafica per la gestione di tutte le funzioni dell'Autore
\item[OrganizerDashboard] :Classe che implementa l'interfaccia grafica per la gestione di tutte le funzioni dell'Organizzatore
\item[SubmitArticleForm] :Classe che implementa l'interfaccia grafica per la sottomissione di articoli
\item[ReviewDashboard] : Classe che implementa l'interfaccia grafica per la gestione delle funzionalità da revisore
\item[ReviewForm]: Classe che implementa l'interfaccia grafica per effettuare la revisione di un articolo
\end{description}


%%% Local Variables:
%%% mode: LaTeX
%%% TeX-master: "main"
%%% End:
