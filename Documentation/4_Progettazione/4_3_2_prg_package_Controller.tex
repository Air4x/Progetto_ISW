\subsubsection{Package Controller}

\label{fig:package_controller_prg_image}
\begin{figure}[ht]
  \centering
  \includegraphics[width=\linewidth]{VisualParadigm/diagramma_prg_controller.png}
  \caption{Package Controller}
  \label{fig:Package Controller}
\end{figure}

Il \texttt{package Controller} contiene le classi che si occupano delle
responsabilità, in particolare dei requisiti funzionali (vedere \ref{sec:requisiti_funzionali}).
In particolare abbiamo:
\begin{description}
\item[UserController] :Classe che implementa la logica di gestione utenti
\item[ConferenceController] :Classe che implementa la logica di gestione delle Conferenzze
\item[ArticleController] :Classe che implementa la logica di gestione degli Articoli
\item[ReviewController] :Classe che implementa la logica di gestione di una revisione
\item[NotificationController] :Classe che estende java.util.TimerTask e implementa la logica di gestione della creazione e invio di email
\end{description}

%%% Local Variables:
%%% mode: LaTeX
%%% TeX-master: "main"
%%% End:
