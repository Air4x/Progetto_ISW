\subsubsection{Package DTO}

\label{sec:package_dto_prg_image}
\begin{figure}[ht]
  \centering
  \includegraphics[width=\linewidth]{VisualParadigm/diagramma_prg_dto.png}
  \caption{Package DTO}
  \label{fig:Package DTO}
\end{figure}

Attraverso l'utilizzo del \texttt{package DTO} si va a mostrare sulla \texttt{GUI} collezioni di elementi le cui dati vengono selezionati. La cui adozione nasce dall'esigenza di evirtare un accoppiamento troppo elevato tra package Entity e i restanti package. Ed è caraterizzato dalle classi:
\begin{description}
\item[ReviewDTO] classe DTO per il trasporto delle informazioni di un revisione
\item[RUserDTO] classe DTO per il trasporto delle informazioni di un utente
\item[ShowActiveConferenceDTO] classe DTO per il trasporto delle informazioni di una conferenzza attiva
\item[ShowArticleDTO] classe DTO per il trasporto delle informazioni di un articolo
\end{description}
%%% Local Variables:
%%% mode: LaTeX
%%% TeX-master: "main"
%%% End:
