\subsubsection{Package Database}
\begin{figure}[ht]
  \centering
  \includegraphics[width=\linewidth]{VisualParadigm/diagramma_prg_database.png}
  \caption{Diagramma delle classi del package Database}
  \label{fig:package_database}
\end{figure}

Il \texttt{package Database} contiene le classi che gestiscono la connesione
alla base dati. In particolare si è deciso di utilizzare un approccio basato
su Data Access Object, così da incapsulare la logica di accesso al database
senza che i livelli superiori se debbano occupare.
Le classi qui definite sono:
\begin{description}
\item[DBManager], è la classe che si occupa di creare la connesione al database, tramite protocollo ODBC
\item[UserDAO], è la classe che si occupa di gestire l'accesso alla tabella Utenti nella base di dati
\item[ConferenceDAO], è la classe che si occupa di gestire l'accesso alla tabella Conferenze nella base di dati
\item[ArticleDAO], è la classe che si occupa di gestire l'accesso alla tabella Articoli nella base di dati
\item[ReviewDAO], è la classe che si occupa di gestire i reviewers nella base di dati
\end{description}

\paragraph{DBManager}
La classe DBManager presenta due metodi:
\begin{description}
\item[Costruttore] Costruttore che inizializza la connesione tra il sistema e la base di dati tramite ODBC
\item[getConnection] Getter per la connesione, utilizzato dalle altre classi
  per poter comunicare con la base di dati, senza avere la responsabilità di
  dover gestire la connesione stessa
\end{description}

\paragraph{UserDAO}
La classe UserDAO presenta i seguenti metodi:
\begin{itemize}
\item getUserRoleByID
\item getUserByID
\item isUserPresentByID
\item isUserPresentByEmail
\item getUserIdByEmail
\item getAllAuthors
\item saveUser
\end{itemize}

\paragraph{ConferenceDAO}
La classe ConferenceDAO presenta i seguenti metodi:
\begin{itemize}
\item getAllConferences
\item getArticlesByConference
\item saveConference
\end{itemize}

\paragraph{ArticleDAO}
La classe ArticleDAO presenta i seguenti metodi:
\begin{itemize}
\item getArticleByAuthor
\item updateArticleStatus
\item getArticleByID
\item saveArticle
\end{itemize}

\paragraph{ReviewDAO}
La classe ReviewDAO presenta i seguenti metodi:
\begin{itemize}
\item assignReviewer
\item getReviewers
\item hasConflictOfInterest
\end{itemize}

\paragraph{Progettazione Database}
% TODO: Riscrivere modello ER e relativa progettazione fisica
Nella figura \ref{fig:modello_er} è presentato il modello ER su cui è basato il
database
\begin{figure}[!ht]
  \centering
  \includegraphics[width=\linewidth]{VisualParadigm/er_finale.png}
  \caption{Modello ER del database}
  \label{fig:modello_er}
\end{figure}

%%% Local Variables:
%%% mode: LaTeX
%%% TeX-master: "main"
%%% End:
