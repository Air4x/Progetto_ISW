
\subsubsection{Package Utility}
\label{sec:prg_utility}
\begin{figure}[!ht]
  \centering
  \includegraphics[width=\linewidth]{VisualParadigm/diagramma_prg_utility.png}
  \caption{Diagramma delle classi del package utility}
  \label{fig:prg_utility}
\end{figure}

Il \texttt{package utility} contiene delle classi di supporto che
implementano funzionalità secondarie utilizzate in posti diversi del
programma, ma che non hanno una reale relazione tra di loro.
Le classi contenute in questo \texttt{package} sono:
\begin{description}
\item[ID], un leggero wrapper per la classe \texttt{java.util.UUID}
  per la generazione di UUID per l'identificazione degli elementi
  gestiti dal sistema
\item[PasswordManager], una classe thread-safe per la gestione dei
  segreti necessari per la gestione delle funzionalità del sistema:
  database e servizio di notifica email
\item[CheckListItem], implementa un item di una lista di checkbox
\end{description}

\paragraph{ID}
La classe ID presenta due metodi principali:
\begin{description}
\item[ID] Costruttore pubblico, serve a convertire un ID dalla sua rappresentazione come stringa alla rappresentazione interna di Java
\item[generate] Metodo statico che implementa la capacità di generare
  un nuovo ID
\end{description}

\paragraph{PasswordManager}
La classe PasswordManager presenta due metodi principali:
\begin{description}
\item[getInstance] Metodo che funziona come costruttore, ma invece di
  creare una nuova istanza della classe, permette di accedere sempre
  alla stessa così da rispettare il Singleton Pattern
\item[get] Metodo che permette di ottenere un segreto dato il suo nome
\end{description}

%%% Local Variables:
%%% mode: LaTeX
%%% TeX-master: "../main"
%%% End:
