\section{Pacchetto Entity}
\label{sec:impl_entity}
\subsection{User}
Classe di dominio astratta, contiene i dati e le funzionalità comuni
tra \texttt{entity.Author} e \texttt{entity.Organizer}
\paragraph{Attributi}
\begin{description}
\item[id] Il codice identificativo dell'utente, è un UUIDv4
\item[name] Il nome dell'utente
\item[lastName] Il cognome dell'utente
\item[email] L'email dell'utente
\item[affiliation] L'università dell'utente
\item[password] Lo SHA256 della password dell'utente
\end{description}

\subsubsection{\texttt{entity.User.User()}}
Istanzia un utente
\paragraph{Parametri}
\begin{itemize}
\item affiliazione
\item name
\item lastName
\item email
\item password
\item id
\end{itemize}

\subsubsection{\texttt{entity.User.User()}}
Istanzia un utente, generando un nuovo Id
\paragraph{Parametri}
\begin{itemize}
  \item affiliazione
  \item email
\item lastName
\item name
\item password
\end{itemize}

\subsubsection{\texttt{entity.User.User()}}
Costruttore di copia per \texttt{entity.User}
\paragraph{Parametri}
\begin{itemize}
\item utente
\end{itemize}

\subsubsection{\texttt{entity.User.getId()}}
\paragraph{Valore di ritorno}
\begin{itemize}
\item l'id dell'utente
\end{itemize}

\subsubsection{\texttt{entity.User.getName()}}
\paragraph{Valore di ritorno}
\begin{itemize}
\item il nome dell'utente
\end{itemize}

\subsubsection{\texttt{entity.User.getLastName()}}
\paragraph{Valore di ritorno}
\begin{itemize}
\item il cognome dell'utente
\end{itemize}

\subsubsection{\texttt{entity.User.getAffiliation()}}
\paragraph{Valore di ritorno}
\begin{itemize}
\item l'affiliazione dell'utente
\end{itemize}

\subsubsection{\texttt{entity.User.getEmail()}}
\paragraph{Valore di ritorno}
\begin{itemize}
\item l'email dell'utente
\end{itemize}

\subsubsection{\texttt{entity.User.getPassword()}}
\paragraph{Valore di ritorno}
\begin{itemize}
\item la password dell'utente
\end{itemize}

\subsubsection{\texttt{entity.User.setAffiliation()}}
Cambia l'affiliazione dell'utente
\paragraph{Parametri}
\begin{itemize}
\item affiliazione
\end{itemize}

\subsubsection{\texttt{entity.User.setEmail()}}
Cambia l'email dell'utente
\paragraph{Parametri}
\begin{itemize}
\item email
\end{itemize}

\subsection{Author}
\paragraph{Attributi}
\begin{description}
\item [role] il ruolo dell'autore
\end{description}

\subsubsection{\texttt{entity.Author.Author()}}
Istanzia un nuovo autore
\paragraph{Parametri}
\begin{itemize}
\item affiliazione
\item email
\item lastName
\item name
\item password
\item id
\end{itemize}

\subsubsection{\texttt{entity.Author.Author()}}
Costruttore di copia per \texttt{Author}
\paragraph{Parametri}
\begin{itemize}
\item autore
\end{itemize}

\subsubsection{\texttt{entity.Author.getRole()}}
\paragraph{Valore di ritorno}
\begin{itemize}
\item Il ruolo dell'autore
\end{itemize}

\subsection{Organizer}
Classe che modella un Organizzatore, estende \texttt{User}
\paragraph{Attributi}
\begin{description}
\item[role] Il ruolo dell'organizzatore
\end{description}

\subsubsection{\texttt{entity.Organizer.Organizer()}}
Istanzia un nuovo organizzatore
\paragraph{Parametri}
\begin{itemize}
\item affiliazione
\item email
\item lastName
\item name
\item password
\item id
\end{itemize}

\subsubsection{\texttt{entity.Organizer.Organizer()}}
Costruttore di copia per \texttt{Organizer}
\paragraph{Parametri}
\begin{itemize}
\item organizer
\end{itemize}

\subsubsection{\texttt{entity.Organizer.getRole()}}
\paragraph{Valore di ritorno}
\begin{itemize}
\item Il ruolo dell'organizzatore
\end{itemize}

\subsection{Article}
\paragraph{Attributi}
\begin{description}
\item[id] Codice identificativo dell'articolo
\item[title] Titolo dell'articolo
\item[abstr]  Abstract dell'articolo
\item[authors] Lista di \texttt{Author} dell'articolo
\end{description}

\subsubsection{\texttt{entity.Article.Article()}}
Istanzia un nuovo articolo
\paragraph{Parametri}
\begin{itemize}
\item id
\item abstr
\item autori
\item titolo
\end{itemize}

\subsubsection{\texttt{entity.Article.Article()}}
Costruttore di copia
\paragraph{Parametri}
\begin{itemize}
\item articolo
\end{itemize}

\subsubsection{\texttt{entity.Article.getAuthors()}}
\paragraph{Valore di ritorno}
\begin{itemize}
\item Lista di autori dell'articolo
\end{itemize}

\subsubsection{\texttt{entity.Article.getAbstr()}}
\paragraph{Valore di ritorno}
\begin{itemize}
\item L'abstract dell'articolo
\end{itemize}

\subsubsection{\texttt{entity.Article.getTitle()}}
\paragraph{Valore di ritorno}
\begin{itemize}
\item Il titolo dell'articolo
\end{itemize}

\subsubsection{\texttt{entity.Article.getId()}}
\paragraph{Valore di ritorno}
\begin{itemize}
\item L'id dell'articolo
\end{itemize}

\subsection{Conference}
\paragraph{Attributi}
\begin{description}
\item[id] Il codice identificativo della conferenze
\item[titolo] Il titolo della conferenza
\item[descrizione] La descrizione della conferenza
\item[scadenza] La scadenza per la sottomissione degli articoli
\item[articoli] La lista di \texttt{Articolo} sottomessi alla conferenza
\end{description}

\subsubsection{\texttt{entity.Conference.Conference()}}
Istanzia una conferenza
\paragraph{Parametri}
\begin{itemize}
\item scadenza
\item titolo
\item descrizione
\item id
\end{itemize}

\subsubsection{\texttt{entity.Conference.Conference()}}
Costruttore di copia per \texttt{Conference}
\paragraph{Parametri}
\begin{itemize}
\item conferenza
\end{itemize}

\subsubsection{\texttt{entity.Conference.getArticles()}}
\paragraph{Valore di ritorno}
\begin{itemize}
\item Lista degli articoli sottomessi alla conferenza
\end{itemize}
\subsubsection{\texttt{entity.Conference.getId()}}
\paragraph{Valore di ritorno}
\begin{itemize}
\item L'id della conferenza
\end{itemize}
\subsubsection{\texttt{entity.Conference.getDescription()}}
\paragraph{Valore di ritorno}
\begin{itemize}
\item La descrizione della conferenza
\end{itemize}
\subsubsection{\texttt{entity.Conference.getDeadline()}}
\paragraph{Valore di ritorno}
\begin{itemize}
\item La scadenza per la sottomissione degli articoli
\end{itemize}
\subsubsection{\texttt{entity.Conference.getTitle()}}
\paragraph{Valore di ritorno}
\begin{itemize}
\item Il titolo della conferenza
\end{itemize}
\subsubsection{\texttt{entity.Conference.nearDeadline()}}
Predicato che verifica se una conferenza è in ``scadenza''
\paragraph{Valore di ritorno}
\begin{itemize}
\item Vero se la conferenza è in scadenza
\item False se la conferenza non  è in scadenza
\end{itemize}

%%% Local Variables:
%%% mode: LaTeX
%%% TeX-master: "main"
%%% End:
