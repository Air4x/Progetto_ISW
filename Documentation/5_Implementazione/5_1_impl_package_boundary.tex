\section{Pacchetto Boundary}
\label{sec:package_boundary}

\subsection{LoginVIew}
Classe utilizzata per la scelta delle opzioni di accesso 
\paragraph{Attributi}
\begin{description}
\item[panel] Istanza di JPanel
\item[loginButton] Istanza di JButton
\item[registerButton] Istanza di JButton
\end{description}

\subsubsection{\texttt{boundary.LoginView.LoginView()}}
Costruttore


\subsection{RegistrationForm}
Classe utilizzata per registrare l'utente e, in caso di successo, accedere alla Dashboard
\paragraph{Attributi}
\begin{description}
\item[txtname] Istanza di JTextField
\item[lblname] Istanza di JLabel
\item[txtlastname] Istanza di JTextField
\item[lbllastname] Istanza di JLabel
\item[txtemail] Istanza di JTextField
\item[lblemail] Istanza di JLabel
\item[passwordField] Istanza di JPasswordField
\item[lblpassword] Istanza di JLabel
\item[txtaffilazione] Istanza di JTextField
\item[lblaffiliazione] Istanza di JLabel
\item[lblruolo] Istanza di JLabel
\item[comboboxruoli] Istanza di JComboBox
\item[registerbutton] Istanza di JButton
\item[userDTO] Istanza di RUserDTO
\item[panel] Istanza di JPanel 
\end{description}

\subsubsection{boundary.RegistrationForm.RegistrationForm()}
Costruttore
\paragraph{Eccezioni}
\begin{description}
\item[SQLException] In caso di errore di interazione con il Database
\end{description}

\subsection{LoginForm}
Classe utilizzata per accedere alla propria DashBoard di un utente precedentemente Registrato
\paragraph{Attributi}
\begin{description}
\item[txtemail] Istanza di JTextField
\item[lblemail] Istanza di JLabel
\item[passwordfield] Istanza di JPasswordField
\item[lblpassword] Istanza di JLabel
\item[loginbutton] Istanza di JButton
\item[panel] Istanza di JPanel
\item[userDTO] Istanza di RUserDTO
\end{description}

\subsubsection{boundary.LoginForm.LoginForm()}
Costruttore
\paragraph{Eccezioni}
\begin{description}
\item[SQLException] In caso di errore di interazione con il Database
\end{description}

\subsection{AuthorDashboard}
Classe Utilizzata per visualizzare tutte le conferenze attive e di poter inoltrare articoli e visualizzarli 
\paragraph{Attributi}
\begin{description}
\item[contentPane] Istanza di JPanel 
\item[scrollActiveConference] Istanza di JScrollPane
\item[scrollSubmittedArticle] Istanza di JScrollPane
\item[listactiveconference] Istanza di JList
\item[listarticlesubmitted] Istanza di JList
\item[lblwelcome] Istanza di JLabel
\item[lblarticlesubmitted] Istanza di JLabel
\item[lblactiveconference] Istanza di JLabel
\end{description}

\subsubsection{boundary.AuthorDashboard.AuthorDashboard()}
Costruttore
\paragraph{Parametri}
\begin{description}
\item[userDTO] RUserDTO
•
\end{description}•
\paragraph{Eccezioni}
\begin{description}
\item[SQLException] In caso di errore di interazione con il Database
\end{description}

\subsection{OrganizerDashboard}
Classe utilizzata per visualizzare le conferenze attive e gestire gli articoli delle suddette conferenze, assegnando per ogni articolo i revisori
\paragraph{Attributi}
\begin{description}
\item[contentPane] Istanza di JPanel
\item[scrollConferenceList] Istanza di JScrollPane
\item[buttonCreateConference] Istanza di JButton
\item[listActiveConference] Istanza di JList
\item[listArticleSubmitted] Istanza di JList
\item[scrollarticleSubmitted] Istanza di JScrollPane
\item[lblwelcome] Istanza di JLabel
\item[lblactiveconference] Istanza di JLabel
\item[lblarticlesubmitted] Istanza di JLabel
\end{description}

\subsubsection{boundary.OrganizerDashboard.OrganizerDashboard()}
Costruttore
\paragraph{Parametri}
\begin{description}
\item[organizer] RUserDTO
•
\end{description}•
\paragraph{Eccezioni}
\begin{description}
\item[SQLException] In caso di errore di interazione con il Database
\end{description}}

\subsection{SubmitArticleForm}
Classe per la compilazione del form per la sottomissione di un articolo alla conferenza selezionata
\paragraph{Attributi}
\begin{description}
\item[lbltitle] Istanza di JLabel
\item[txttitle] Istanza di JTextField
\item[lblabstract] Istanza di JLabel
\item[txtareaabstract] Istanza di JTextArea
\item[lblcoauthors] Istanza di JLabel
\item[txtcoauthors] Istanza di JTextField
\item[contentPane] Istanza di JPanel
\item[buttonSubmit] Istanza di JButton
\item[buttonBack] Istanza di JButton
\end{description}•}

\subsubsection{boundary.SubmitArticleForm.SubmitArticleForm()}
Costruttore
\paragraph{Parametri}
\begin{description}
\item[userDTO] RUserDTO
\item[conferenceID] ID
• 
\end{description}•
\paragraph{Eccezioni}
\begin{description}
\item[SQLException] In caso di errore di interazione con il Database

\end{description}•}

\subsection{AssignReviewersView}
Classe che permette ad un Organizzatore di assegnare i revisori ad un determinato articolo
\paragraph{Attributi}
\begin{description}
\item[contentPane] Istanza di JPanel
\item[lblrevisori] Istanza di JLabel
\item[buttonAssignReviewers] Istanza di JButton
\item[reviewer1]Istanza di JComboBox
\item[reviewer2]Istanza di JComboBox
\item[reviewer3] Istanza di JComboBox
\item[selected1] Istanza di PossibleReviewDTO
\item[selected2] Istanza di PossibleReviewDTO
\item[selected3] Istanza di PossibleReviewDTO
\end{description}•}

\subsubsection{boundary.AssignReviewersView.AssignReviewersView()}
Costruttore
\paragraph{Parametri}
\begin{description}
\item[article] ShowArticleDTO
\end{description}•
\paragraph{Eccezioni}
\begin{description}
\item[SQLException]  In caso di errore di interazione con il Database
•
\end{description}•

\subsection{CreateConferenceForm}
Classe per la creazione da parte di un organizzatore di una conferenza
\paragraph{Attributi}
\begin{description}
\item[contentPane] Istanza di JPanel
\item[lbltitle] Istanza di JLabel
\item[txttitle] Istanza di JTextField
\item[lbldescription] Istanza di JLabel
\item[txtareadescription] Istanza di JTextArea
\item[lblduedate] Istanza di JLabel
\item[txtduedate] Istanza di JTextField
\item[buttonCreateConference] Istanza di JButton
\item[buttonBack] Istanza di JButton

\end{description}•

\subsubsection{boundary.CreateConferenceForm.CreateConferenceForm()}
Costruttore
\paragraph{Parametri}
\begin{description}
\item[organizer] RUserDTO
\end{description}•
\paragraph{Eccezioni}
\begin{description}
\item[SQLException]  In caso di errore di interazione con il Database
•
\end{description}• 














%%% Local Variables:
%%% mode: LaTeX
%%% TeX-master: "main"
%%% End:
