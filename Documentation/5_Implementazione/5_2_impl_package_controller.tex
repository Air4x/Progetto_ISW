\section{Pacchetto Controller}
\label{sec:package_controller}

\subsection{ArticleController}
Classe utilizzata per la gestione degli Articoli
\paragraph{Attributi}
\begin{description}
\item[art-dao] Instanza di ArticleDAO
\item[user-dao] Instanza di UserDAO
\item[conf-dao] Instanza di ConferenceDAO
\end{description}

\subsubsection{\texttt{controller.ArticleController.ArticleController()}}
Costruttore
\paragraph{Eccezioni}
\begin{description}
\item[SQLException] In caso di errore di interazione con il DataBase
\end{description}

\subsubsection{\texttt{controller.ArticleController.submitArticle()}}
Metodo per la sottomisione di un articolo a una conferenza
\paragraph{Parametri}
\begin{description}
\item article-titolo
\item article-abstrct
\item article-autori
\item id-conf
\end{description}
\paragraph{Eccezioni}
\begin{description}
\item[SQLException] In caso di errore di interazione con il DataBase
\end{description}
\paragraph{Valore di ritorno}
Oggetto boolean che va a indicare se la sottomissione di un articoloè avvenuta con successo
\begin{description}
\item [Vero] l'articolo è stato sottomesso corettamente
\item [Falso] non c'è nessuna conferenza con l'id indicato
\end{description}

\subsubsection{\texttt{controller.ArticleController.getArticleByAuthor()}}
Metodo per ottenere una lista di articoli
\paragraph{Parametri}
\begin{description}
\item authorID
\end{description}
\paragraph{Eccezioni}
\begin{description}
\item[SQLException] In caso di errore di interazione con il DataBase
\end{description}
\paragraph{Valore di ritorno}
Un array list di ShowArticleDTO ottenuto attraverso l'ID dell'autore degli articoli



\subsection{ConferenceController}
\paragraph{Attributi}
\begin{description}
\item[user-dao] Instanza di  UserDAO
\item[conf-dao] Instanza di ConferenceDAO
\end{description}

\subsubsection{\texttt{controller.ConferenceController.ConferenceController()}}
Costruttore
\paragraph{Eccezioni}
\begin{description}
\item[SQLException] In caso di errore di interazione con il DataBase
\end{description}

\subsubsection{\texttt{controller.ConferenceController.createConference()}}
Metodo per la creazione di una conferenzza
\paragraph{Parametri}
\begin{description}
\item scadenza
\item title
\item descr
\item id
\item org
\end{description}
\paragraph{Eccezioni}
\begin{description}
\item[SQLException] In caso di errore di interazione con il DataBase
\end{description}
\paragraph{Valore di ritorno}
Valore booleano che indica se la conferenza è stata creata con successo
\begin{description}
\item [Vero] Conferenza creata con successo
\item [Falso] Errore nella creazione della conferenza
\end{description}

\subsubsection{\texttt{controller.ConferenceController.getActiveConferences()}}
Metodo per ottenere una lista di conferenzze attive [NON ANCORA SCADUTE]
\paragraph{Eccezioni}
\begin{description}
\item[SQLException] In caso di errore di interazione con il DataBase
\end{description}
\paragraph{Valore di ritorno}
Un array list di ShowActiveConferenceDTO che indicano le conferenze attive

\subsubsection{\texttt{controller.ConferenceController.getArticlesByConference()}}
Metodo per ottenere una lista di articoli
\paragraph{Parametri}
\begin{description}
\item ID-conference
\end{description}
\paragraph{Eccezioni}
\begin{description}
\item[SQLException] In caso di errore di interazione con il DataBase
\end{description}
\paragraph{Valore di ritorno}
Un array list di ShowArticleDTO che indicano gli articoli della conferenza



\subsection{NotificationController}
\paragraph{Attributi}
\begin{description}
\item[conf-dao] Instanza di ConferenceDAO
\item[user-dao] Instanza di UserDAO
\end{description}

\subsubsection{\texttt{controller.NotificationController.NotificationController()}}
Costruttore
\paragraph{Eccezioni}
\begin{description}
\item[SQLException] In caso di errore di interazione con il DataBase
\end{description}

\subsubsection{\texttt{controller.NotificationController.invioNotifiche()}}
Metodo utilizzato per l'invio di email
\paragraph{Eccezioni}
\begin{description}
\item[SQLException] In caso di errore di interazione con il DataBase
\item[MessagingException] Messaggio di errore in caso di eccezione
\end{description}


\subsubsection{\texttt{controller.NotificationController.creatMessage()}}
Metodo per la creazione del messaggio da inviare
\paragraph{Parametri}
\begin{description}
\item aut-name
\item aut-lastname
\item conf-title
\end{description}
\paragraph{Valore di ritorno}
Stringa che rapresenta il messagio da inviare

\subsubsection{\texttt{controller.NotificationController.sendEmail()}}
Metodo utilizzo per la configurazione di un host per l'invio delle email
\paragraph{Parametri}
\begin{description}
\item email-d
\item msg
\item subject
\end{description}
\paragraph{Eccezioni}
\begin{description}
\item[SQLException] In caso di errore di interazione con il DataBase
\item[MessagingException] Messaggio di errore in caso di eccezione
\end{description}


\subsection{ReviewController}
\paragraph{Attributi}
\begin{description}
\item[reviewer-dao] Instanza di ReviewDAO
\item[user-dao] Instanza di UserDAO
\item[article-dao] Instanza di ArticleDAO
\end{description}

\subsubsection{\texttt{controller.ReviewController.ReviewController()}}
Costruttore
\paragraph{Eccezioni}
\begin{description}
\item[SQLException] In caso di errore di interazione con il DataBase
\end{description}


\subsubsection{\texttt{controller.ReviewController.createReview()}}
 Metodo per creare una revisione per ogni revisore passato come parametro 
\paragraph{Parametri}
\begin{description}
\item reviewers
\item articles
\end{description}
\paragraph{Eccezioni}
\begin{description}
\item[SQLException] In caso di errore di interazione con il DataBase
\end{description}
\paragraph{Valore di ritorno}
Valore Booleano che indica se la revisione è stata creata con successo
\begin{description}
\item[Vero] Se la revisione è stata creata correttamente
\item[Falso] Se la lista di revisori è nulla, maggiore di 3 o pari a 0 elementi, o in caso non vi sia nessun articolo legato all'id dato
\end{description}



\subsubsection{\texttt{controller.ReviewController.getPossibleReviewers()}}
Metodo che restituisce la lista di revisori che non hanno conflitti di interesse con l'articolo
\paragraph{Parametri}
\begin{description}
\item articleId
\end{description}
\paragraph{Eccezioni}
\begin{description}
\item[SQLException] In caso di errore di interazione con il DataBase
\end{description}
\paragraph{Valore di ritorno}
Un arraylist di RUser che possono essere scelti come revisori per l'articolo senza conplitti di interesse


\subsubsection{\texttt{controller.ReviewController.updateReview()}}
Metodo che permette di aggiornare una revisione
\paragraph{Parametri}
\begin{description}
\item final-review
\item new-score
\item new-result
\end{description}
\paragraph{Eccezioni}
\begin{description}
\item[SQLException] In caso di errore di interazione con il DataBase
\end{description}
\paragraph{Valore di ritorno}
Un oggetto reviewDTO che rappresenta la revisione aggiornata


\subsubsection{\texttt{controller.ReviewController.getAllReviewsByReviewer()}}
Metodo che permette di ottenere la lista di tutte le revisioni di un datorevisore
\paragraph{Parametri}
\begin{description}
\item reviewer
\end{description}
\paragraph{Eccezioni}
\begin{description}
\item[SQLException] In caso di errore di interazione con il DataBase
\end{description}
\paragraph{Valore di ritorno}
Un arraylist di reviewDTO che rappresenta la lista di tutte le revisioni di un dato revisore


\subsubsection{\texttt{controller.ReviewController.getAllReviewsByArticle()}}
Metodo che permette di ottenere la lista di tutte le revisioni di un dato articolo
\paragraph{Parametri}
\begin{description}
\item articleId
\end{description}
\paragraph{Eccezioni}
\begin{description}
\item[SQLException] In caso di errore di interazione con il DataBase
\end{description}
\paragraph{Valore di ritorno}
Un arraylist di reviewDTO che rappresenta la lista di tutte le revisioni di un dato articolo

\subsubsection{\texttt{controller.ReviewController.checkReviewsCompletion()}}
Metodo che permette di ottenere la lista di tutte le revisioni di un dato articoloMetodo che controlla se un articolo ha ricevuto almeno 2 revisioni con esitopositivo o negativo e aggiorna lo stato dell'articolo di conseguenza
\paragraph{Parametri}
\begin{description}
\item articleId
\end{description}
\paragraph{Eccezioni}
\begin{description}
\item[SQLException] In caso di errore di interazione con il DataBase
\end{description}

\subsection{UserController}
\paragraph{Attributi}
\begin{description}
\item[user-dao] Instanza di UserDAO
\end{description}

\subsubsection{\texttt{controller.UserController.}}
Costruttore
\paragraph{Eccezioni}
\begin{description}
\item[SQLException] In caso di errore di interazione con il DataBase
\end{description}

\subsubsection{\texttt{controller.UserController.registerUser()}}
Metodo per la registarzione di un untente
\paragraph{Parametri}
\begin{description}
\item affiliazione
\item email
\item lastname
\item name
\item password
\item ruole
\end{description}
\paragraph{Eccezioni}
\begin{description}
\item[SQLException] In caso di errore di interazione con il DataBase
\end{description}
\paragraph{Valore di ritorno}
Un oggetto RUser che rappresenta l'unbtente appena registrato

\subsubsection{\texttt{controller.UserController.login}}
Metodo per l'accesso di un utente
\paragraph{Parametri}
\begin{description}
\item email
\item password
\end{description}
\paragraph{Eccezioni}
\begin{description}
\item[SQLException] In caso di errore di interazione con il DataBase
\end{description}
\paragraph{Valore di ritorno}
Un oggetto RUserDTO che rappresenta l'uttente che ha appena fatto il login


\subsubsection{\texttt{controller.UserController.getRAuthorBYEmail()}}
Metodo per ottenere un autore mediante email
\paragraph{Parametri}
\begin{description}
\item email
\end{description}
\paragraph{Eccezioni}
\begin{description}
\item[SQLException] In caso di errore di interazione con il DataBase
\end{description}
\paragraph{Valore di ritorno}
Un oggetto RUserDTO

\subsubsection{\texttt{controller.UserController.getCooAuthors()}}
Metodo per ottenere una lista di coo-autori mediante email
\paragraph{Parametri}
\begin{description}
\item email
\end{description}
\paragraph{Eccezioni}
\begin{description}
\item[SQLException] In caso di errore di interazione con il DataBase
\end{description}
\paragraph{Valore di ritorno}
Un oggetto ArrayList di RUserDTO


%%% Local Variables:
%%% mode: LaTeX
%%% TeX-master: "main"
%%% End:
