\section{Pacchetto Controller}
\label{sec:package_controller}

\subsection{ArticleController}
Classe utilizzata per la gestione degli Articoli
\paragraph{Attributi}
\begin{description}
\item[art-dao] Instanza di ArticleDAO
\item[user-dao] Instanza di UserDAO
\item[conf-dao] Instanza di ConferenceDAO
\end{description}

\subsubsection{\texttt{controller.ArticleController.ArticleController()}}
Costruttore
\paragraph{Eccezioni}
\begin{description}
\item[SQLException] In caso di errore di interazione con il DataBase
\end{description}

\subsubsection{\texttt{controller.ArticleController.submitArticle()}}
Metodo per la sottomisione di un articolo a una conferenza
\paragraph{Parametri}
\begin{description}
\item article-titolo
\item article-abstrct
\item article-autori
\item id-conf
\end{description}
\paragraph{Eccezioni}
\begin{description}
\item[SQLException] In caso di errore di interazione con il DataBase
\end{description}
\paragraph{Valore di ritorno}
Oggetto boolean che va a indicare se la sottomissione di un articoloè avvenuta con successo
\begin{description}
\item [Vero] l'articolo è stato sottomesso corettamente
\item [Falso] non c'è nessuna conferenza con l'id indicato
\end{description}

\subsubsection{\texttt{controller.ArticleController.getArticleByAuthor()}}
Metodo per ottenere una lista di articoli
\paragraph{Parametri}
\begin{description}
\item authorID
\end{description}
\paragraph{Eccezioni}
\begin{description}
\item[SQLException] In caso di errore di interazione con il DataBase
\end{description}
\paragraph{Valore di ritorno}
Un array list di ShowArticleDTO ottenuto attraverso l'ID dell'autore degli articoli



\subsection{ConferenceController}
\paragraph{Attributi}
\begin{description}
\item[user-dao] Instanza di  UserDAO
\item[conf-dao] Instanza di ConferenceDAO
\end{description}

\subsubsection{\texttt{controller.ConferenceController.ConferenceController()}}
Costruttore
\paragraph{Eccezioni}
\begin{description}
\item[SQLException] In caso di errore di interazione con il DataBase
\end{description}

\subsubsection{\texttt{controller.ConferenceController.createConference()}}
Metodo per la creazione di una conferenzza
\paragraph{Parametri}
\begin{description}
\item scadenza
\item title
\item descr
\item id
\item org
\end{description}
\paragraph{Eccezioni}
\begin{description}
\item[SQLException] In caso di errore di interazione con il DataBase
\end{description}
\paragraph{Valore di ritorno}
Valore booleano che indica se la conferenza è stata creata con successo
\begin{description}
\item [Vero] Conferenza creata con successo
\item [Falso] Errore nella creazione della conferenza
\end{description}

\subsubsection{\texttt{controller.ConferenceController.getActiveConferences()}}
Metodo per ottenere una lista di conferenzze attive [NON ANCORA SCADUTE]
\paragraph{Eccezioni}
\begin{description}
\item[SQLException] In caso di errore di interazione con il DataBase
\end{description}
\paragraph{Valore di ritorno}
Un array list di ShowActiveConferenceDTO che indicano le conferenze attive

\subsubsection{\texttt{controller.ConferenceController.getArticlesByConference()}}
Metodo per ottenere una lista di articoli
\paragraph{Parametri}
\begin{description}
\item ID-conference
\end{description}
\paragraph{Eccezioni}
\begin{description}
\item[SQLException] In caso di errore di interazione con il DataBase
\end{description}
\paragraph{Valore di ritorno}
Un array list di ShowArticleDTO che indicano gli articoli della conferenza



\subsection{NotificationController}
\paragraph{Attributi}
\begin{description}
\item[conf-dao] Instanza di ConferenceDAO
\item[user-dao] Instanza di UserDAO
\end{description}

\subsubsection{\texttt{controller.NotificationController.NotificationController()}}
Costruttore
\paragraph{Eccezioni}
\begin{description}
\item[SQLException] In caso di errore di interazione con il DataBase
\end{description}

\subsubsection{\texttt{controller.NotificationController.invioNotifiche()}}
Metodo utilizzato per l'invio di email
\paragraph{Eccezioni}
\begin{description}
\item[SQLException] In caso di errore di interazione con il DataBase
\item[MessagingException] Messaggio di errore in caso di eccezione
\end{description}


\subsubsection{\texttt{controller.NotificationController.creatMessage()}}
Metodo per la creazione del messaggio da inviare
\paragraph{Parametri}
\begin{description}
\item aut-name
\item aut-lastname
\item conf-title
\end{description}
\paragraph{Valore di ritorno}
Stringa che rapresenta il messagio da inviare

\subsubsection{\texttt{controller.NotificationController.sendEmail()}}
Metodo utilizzo per la configurazione di un host per l'invio delle email
\paragraph{Parametri}
\begin{description}
\item email-d
\item conf-title
\item msg
\end{description}
\paragraph{Eccezioni}
\begin{description}
\item[SQLException] In caso di errore di interazione con il DataBase
\item[MessagingException] Messaggio di errore in caso di eccezione
\end{description}


\subsection{ReviewController}
\paragraph{Attributi}
\begin{description}
\item[reviewer-dao] Instanza di ReviewDAO
\item[user-dao] Instanza di UserDAO
\item[article-dao] Instanza di ArticleDAO
\end{description}

\subsubsection{\texttt{controller.ReviewController.ReviewController()}}
Costruttore
\paragraph{Eccezioni}
\begin{description}
\item[SQLException] In caso di errore di interazione con il DataBase
\end{description}


\subsubsection{\texttt{controller.ReviewController.assignReviewer()}}
Metodo per l'assegnazion di revisore
\paragraph{Parametri}
\begin{description}
\item articleID
\item list-reviewer-selected
\end{description}
\paragraph{Eccezioni}
\begin{description}
\item[SQLException] In caso di errore di interazione con il DataBase
\end{description}
\paragraph{Valore di ritorno}
Valore Booleano che indica se l'assegnazione di revisore è avvenuta con successo
\begin{description}
\item[Vero] Se i revisori sono stati assegnati correttamente 
\item[Falso] Se la lista risulta nulla, maggiore di 3 o pari 0 elementi o in caso che non vi e nessu articolo legato all'id dato
\end{description}



\subsubsection{\texttt{controller.ReviewController.getListReviewer()}}
Metodo per ottenere una lista di possibili revisori
\paragraph{Parametri}
\begin{description}
\item articleID
\end{description}
\paragraph{Eccezioni}
\begin{description}
\item[SQLException] In caso di errore di interazione con il DataBase
\end{description}
\paragraph{Valore di ritorno}
Un arraylist di PossibleReviewDTO che si possono assegnare come revisori


\subsubsection{\texttt{controller.ArticleController.updateArticleStatus()}}
Metodo per aggiorna lo status di un articolo
\paragraph{Parametri}
\begin{description}
\item id-article
\item status
\end{description}
\paragraph{Eccezioni}
\begin{description}
\item[SQLException] In caso di errore di interazione con il DataBase
\end{description}
\paragraph{Valore di ritorno}
Valore Booleano che indica se l'aggiornamento dello stato è avvenuto con successo
\begin{description}
\item[Vero] se lo stato del'articolo è sato corretamente aggiornato
\item[Falso] articolo non trovato
\end{description}



\subsection{UserController}
\paragraph{Attributi}
\begin{description}
\item[user-dao] Instanza di UserDAO
\end{description}

\subsubsection{\texttt{controller.UserController.}}
Costruttore
\paragraph{Eccezioni}
\begin{description}
\item[SQLException] In caso di errore di interazione con il DataBase
\end{description}

\subsubsection{\texttt{controller.UserController.registerUser()}}
Metodo per la registarzione di un untente
\paragraph{Parametri}
\begin{description}
\item affiliazione
\item email
\item lastname
\item name
\item password
\item ruole
\end{description}
\paragraph{Eccezioni}
\begin{description}
\item[SQLException] In caso di errore di interazione con il DataBase
\end{description}
\paragraph{Valore di ritorno}
Un oggetto RUser che rappresenta l'unbtente appena registrato

\subsubsection{\texttt{controller.UserController.}}

\paragraph{Parametri}
\begin{description}
\item email
\item password
\end{description}
\paragraph{Eccezioni}
\begin{description}
\item[SQLException] In caso di errore di interazione con il DataBase
\end{description}
\paragraph{Valore di ritorno}
Un oggetto RUserDTO che rappresenta l'uttente che ha appena fatto il login

%%% Local Variables:
%%% mode: LaTeX
%%% TeX-master: "main"
%%% End:
