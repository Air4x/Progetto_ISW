\subsection{Package Controller}
\label{sec:test_strutturale_package_controller}

\subsubsection{Creazione di Nuova Conferenzza: createConference}
Il metodo createConference serve a creare una nuova conferenza, verificando che non esista già, che la data di scadenza sia nel futuro e che l'organizzatore sia valido
\begin{lstlisting}
public boolean createConference (LocalDate scadenza, String title, String descr, ID id, RUserDTO org) throws SQLException { 
        LocalDate today = LocalDate.now(); [1]
        Date deadline = Date.valueOf(scadenza); [2]
        if(conf_dao.isConferencePresentByID(id)); [3]
            System.out.println("Conference Already Exists"); [4]
            return false; [5]
        }else if(today.isAfter(scadenza)){ [6]
            System.out.println("Scadenza is before today"); [7]
            return false; [8]
        }else if(org != null && user_dao.isUserPresentByID(org.getId())==false || user-dao.getUserByEmail(org.getEmail()).getRole()!="organizzatore") [9]
            System.out.println("Organizzarore is not Found"); [10]
            return false; [11]
        }
        Conference new_conference = new Conference(deadline,title,descr,id,org.getId()); [12]
        conf_dao.saveConference(new-conference); [13]
        System.out.println("Conference is created"); [14]
        return true; [15]
}
\end{lstlisting}
\paragraph{Control Flow Grap}
\begin{figure}[ht]
  \centering
  \includegraphics[width=\linewidth]{VisualParadigm/test_createConference.png}
  \caption{Control Flow Grap di createConference}
  \label{fig:cfg_create_conference }
\end{figure}
\paragraph{Numero Ciclomatico}
\begin{itemize}
\item[.] numero di regioni chiuse del grafo +1 = 3 + 1= 4 
\item[.] numero di nodi predicati + 1 = 3 (3,6,9) +1= 4
\item[.] numero di archi - numero di nodi + 2 = 17 - 15 + 2 = 4 
\end{itemize}
\paragraph{Cammini}
\begin{itemize}
\item[Percoso 1: ] 1 - 2 - 3 - 6 - 9 - 12 - 13 - 14 - 15 
\item[Percoso 2: ] 1 - 2 - 3 -4 - 5
\item[Percoso 3: ] 1 - 2 - 3 - 6 - 7 - 8
\item[Percoso 4: ] 1 - 2 - 3 - 6 - 9 - 10 - 11
\end{itemize}
\paragraph{Casi di Test}
\begin{itemize}
\item[Caso n.1 :] Creazione avenuta con successo
\begin{itemize}
\item[Obiettivo:] Verificare il percorso 1
\item[Condizione:] le condizioni:
\begin{itemize}
\item[.] conf-dao.isConferencePresentByID() risulta false
\item[.] today.isAfter() risulta false
\item[.] user-dao.isUserPresentByID() risulti false o user-dao.getUserByEmail(org.getEmail()).getRole() risulta uguale ad organizzatore
\end{itemize}
\item[Input:] Fornire un id non esistente, una scadenza nel futuro e un org che esiste e ha il ruolo di "organizzatore"
\item[Risultato:] Il metodo deve restituire true e viene invocata il metodo conf-dao.saveConference()
\end{itemize}
\item[Caso n. 2:] Conferenza esistente
\begin{itemize}
\item[Obiettivo:]  Verificare il percorso 2
\item[Condizione:]  Dove le condizioni:
\begin{itemize}
\item[.] conf-dao.isConferencePresentByID() risulta true
\end{itemize}
\item[Input:] Fornire un id che già esiste nel database
\item[Risultato:] Il metodo ristituisce false e sul terminale viene stampato "Conference Already Exists"
\end{itemize}
\item[Caso n. 3:] Scadenza non valida
\begin{itemize}
\item[Obiettivo:]  Verificare il percorso 3
\item[Condizione:] Dove le condizioni:
\begin{itemize}
\item[.] conf-dao.isConferencePresentByID() risulta false
\item[.] today.isAfter() risulta true
\end{itemize}
\item[Input:] Fornire un id non esistente e una scadenza nel passato
\item[Risultato:] Il metodo ristituisce false e sul terminale viene stampato "Scadenza is after today"
\end{itemize}
\item[Caso n. 4:]  Organizzatore non esistente
\begin{itemize}
\item[Obiettivo:]   Verificare il percorso 4
\item[Condizione:]  Dove le condizioni:
\begin{itemize}
\item[.] conf-dao.isConferencePresentByID() risulta false
\item[.] today.isAfter() risulta false
\item[.] user-dao.isUserPresentByID() risulti true o user-dao.getUserByEmail(org.getEmail()).getRole() risulta diverso da organizzatore
\end{itemize}
\item[Input:] Vi sono 2 opzioni:
\begin{itemize}
\item[.] Un id di un utente che non esiste nel database
\item[.] IL ruolo dell'utente è diverso da organizzatore
\end{itemize}
\item[Risultato:] Il metodo ristituisce false e sul terminale viene stampato "Organizzarore is not Found"
\end{itemize}
\end{itemize}

\subsubsection{Sottomisioni di un Articolo: submitArticle}
Il metodo submitArticle gestisce la sottomissione di un articolo a una conferenza, con controlli sulla conferenza, la sua scadenza e la lista degli autori.
\begin{lstlisting}
public boolean submitArticle(String article_titolo, String article_abstract, ArrayList<RUserDTO> article_autori, ID id_conf) throws SQLException { 
        ID article_id = ID.generate(); [1]
        ArrayList<Author> authors_list = new ArrayList<>(); [2]
        if (this.conf_dao.isConferencePresentByID(id_conf)==false){  [3]
            System.out.println("Conference not found"); [4]
            return false; [5]
        }
        if(this.conf_dao.getConferenceByID(id_conf).getDeadline().before(Date.valueOf(LocalDate.now()))){ [6]
            System.out.println("The selected conference is expired"); [7]
            return false; [8]
        }
        if(article_autori.isEmpty()){ [9]
            System.out.println("List of authors is empty"); [10] 
            return false; [11]
        }
        for (RUserDTO fake_user : article_autori) { [12]
             if(user_dao.isUserPresentByEmail(fake_user.getEmail()) == true && user_dao.getUserByEmail(fake_user.getEmail()).getRole() == "autore" && fake_user != null) { [13]
                authors_list.add((Author) user_dao.getUserByEmail(fake_user.getEmail()));[14]
            }
        }
	if(authors_list.isEmpty()){[15]
            System.out.println("List of authors is empty"); [16]
            return false;[17]
        }
        Article art = new Article(article_id, article_abstract, authors_list, article_titolo); [18]
        this.conf_dao.getConferenceByID(id_conf).getArticles().add(art); [19]
        art_dao.saveArticle(art); [20]
        return true; [21]
   } 
\end{lstlisting} 
\paragraph{Control Flow Grap}
\paragraph{Numero Ciclomatico}
\begin{itemize}
\item[.] numero di regioni chiuse del grafo + 1= 6 +1 = 7
\item[.] numero di nodi predicati +1 = 6 (2,4,6,8,9,11) +1 = 7
\item[.] numero di archi - numero di nodi + 2 = 26 - 21 + 2 = 6 + 2 = 7  
\end{itemize}
\paragraph{Cammini}
\begin{itemize}
\item[Percorso n. 1: ]  1 - 2 - 3 - 6 - 9 - 12 - 13 - 14 - 15 - 18 - 19 - 20 - 21
\item[Percorso n. 2: ] 1 - 2 - 3 - 4 -5
\item[Percorso n. 3: ] 1 - 2 - 3 - 6 - 7 - 8
\item[Percorso n. 4: ] 1 - 2 - 3 - 6 - 9 - 10 - 11  
\item[Percorso n. 5: ] 1 - 2 - 3 - 6 - 9 - 12 - 13 - 14 - 15 - 16 -17
\end{itemize}
\paragraph{Casi di Test}
\begin{figure}[ht]
  \centering
  \includegraphics[width=\linewidth]{VisualParadigm/test_submitArticle.png}
  \caption{Control Flow Grap di submitArticle}
  \label{fig:cfg_submitArticle }
\end{figure}
\begin{itemize}
\item[Caso n. 1 :] Consegnato Articolo con successo [tutti autorri validi max 3]
\begin{itemize}
\item[Obiettivo:] Verificare il percorso 1
\item[Condizione:] le condizioni:
\begin{itemize}
\item[.]  this.conf-dao.isConferencePresentByID() risulti true
\item[.]  this.conf-dao.getConferenceByID(id-conf).getDeadline().before(Date.valueOf(LocalDate.now())) risulti false
\item[.] article-autori.isEmpty() risulti false
\item[.] user-dao.isUserPresentByEmail(fake-user.getEmail()) risulti true
\item[.] user-dao.getUserByEmail(fake-user.getEmail()).getRole() risulti uguale "autore" 
\item[.] authors-list.isEmpty() e false
\item[.]  fake-user risulti diverso da null
\end{itemize}
\item[Input:] Fornire una id di conferenza valida non scaduta e una lista article-autori validi non vuota 
\item[Risultato:] return true
\end{itemize}
\item[Caso n. 2 :] Conferenza inesistente
\begin{itemize}
\item[Obiettivo:] Verificare il percorso 2
\item[Condizione:] le condizioni:
\begin{itemize}
\item[.] this.conf-dao.isConferencePresentByID() risulti false
\end{itemize}
\item[Input:] Fornire una id di conferenza non valida
\item[Risultato:] return false
\end{itemize}
\item[Caso n. 3 :] Conferenza scaduta
\begin{itemize}
\item[Obiettivo:] Verificare il percorso 3
\item[Condizione:] le condizioni:
\begin{itemize}
\item[.]  this.conf-dao.isConferencePresentByID() risulti true
\item[.]  this.conf-dao.getConferenceByID(id-conf).getDeadline().before(Date.valueOf(LocalDate.now())) risulti true 
\end{itemize}
\item[Input:] Fornire una id di conferenza valida ma scaduta
\item[Risultato:] return false
\end{itemize}
\item[Caso n. 4 :] Lista autori vuota
\begin{itemize}
\item[Obiettivo:] Verificare il percorso  4
\item[Condizione:] le condizioni:
\begin{itemize}
\item[.]  this.conf-dao.isConferencePresentByID() risulti true
\item[.]  this.conf-dao.getConferenceByID(id-conf).getDeadline().before(Date.valueOf(LocalDate.now())) risulti false
\item[.] article-autori.isEmpty() risulti true
\end{itemize}
\item[Input:] Fornire una id di conferenza valida non scaduta e una lista article-autori vuota 
\item[Risultato:] return false
\end{itemize}
\item[Caso n. 5 :] lista autoti dto vuota
\begin{itemize}
\item[Obiettivo:] Verificare il percorso 5
\item[Condizione:] le condizioni: 
\begin{itemize}
\item[.]  this.conf-dao.isConferencePresentByID() risulti true
\item[.]  this.conf-dao.getConferenceByID(id-conf).getDeadline().before(Date.valueOf(LocalDate.now())) risulti false
\item[.] article-autori.isEmpty() risulti false
\item[.] user-dao.isUserPresentByEmail(fake-user.getEmail()) risulti false o user-dao.getUserByEmail(fake-user.getEmail()).getRole() risulti diverso da "autore" o  fake-user risulti uguale a null
\item[.] authors-list.isEmpty() e true
\end{itemize}
\item[Input:]  Fornire una id di conferenza valida non scaduta e una lista article-autori con autori non validi
\item[Risultato:] return true. La lista authors-list risulterà vuota, ma il metodo continuerà e salverà l'articolo con una lista di autori vuota.
\end{itemize}
\end{itemize}



%%% Local Variables:
%%% mode: LaTeX
%%% TeX-master: "main"
%%% End: