\subsection{Package Entity}
\label{sec:test_strutturale_package_entity}

\subsubsection{Vedere se un conferenza sta per scadere: nearDeadline}
\begin{lstlisting}
public boolean nearDeadline() {
	int giornoScadenza = this.getDeadline().toLocalDate().getDayOfYear(); [1]
	int giornoNow = LocalDate.now().getDayOfYear(); [2]
        if((giornoScadenza - giornoNow) <= 5 && (giornoScadenza - giornoNow) >= 0) { [3]
	    return true; [4]
        } else { return false; } [5]
    }
\end{lstlisting}
\paragraph{Control Flow Grap}
\begin{figure}[ht]
  \centering
  \includegraphics[width=\linewidth]{VisualParadigm/test_nearDeadline.png}
  \caption{Control Flow Grap di nearDeadline}
  \label{fig:cfg_nearDeadline}
\end{figure}
\paragraph{Numero Ciclomatico}
\begin{itemize}
\item[.] numero di regioni chiuse del grafo +1 = 2 + 1= 3 
\item[.] numero di nodi predicati + 1 = 2 (2,3) +1= 3
\item[.] numero di archi - numero di nodi + 2 = 4 - 5 + 2 = 1 
\end{itemize}
\paragraph{Cammini}
\begin{itemize}
\item[Percorso 1:] 1 - 2 - 3
\item[Percorso 2:] 1 - 2 - 4
\end{itemize}
\paragraph{Casi di Test}
\begin{itemize}
\item[Casi n. 1:]
\begin{itemize}
\item[Obiettivo:] Verificare il percorso 
\item[Condizioni:] Le condizioni sono:
\begin{itemize}
\item[.] (giornoScadenza - giornoNow) <= 5 e  (giornoScadenza - giornoNow ristituisca true
\end{itemize}
\item[Input:]  Che la data della conferenza abbia un scadenza distante 5 giorni dal giorno attuale
\item[Risultato:] return false
\end{itemize}
\item[Casi n. 2:]
\begin{itemize}
\item[Obiettivo:] Verificare il percorso 
\item[Condizioni:] Le condizioni sono:
\begin{itemize}
\item[.] (giornoScadenza - giornoNow) <= 5 e (giornoScadenza - giornoNow ristituisca false
\end{itemize}
\item[Input:] Che la data della conferenza abbia un scadenza distante più distante di 5 giorni dal giorno attuale
\item[Risultato:] return false
\end{itemize}
\end{itemize}





%%% Local Variables:
%%% mode: LaTeX
%%% TeX-master: "main"
%%% End: