\subsection{Package Database}
\label{sec:test_strutturale_package_database}

\subsubsection{Registrazione di un utente: saveUser()}
\begin{lstlisting}
   public void saveUser(User user) throws SQLException {
        if (user.getRole().equals("autore")) { [1]
            Author a = (Author) user; [2]
            String sql = "Insert Into Utenti(ID, NOME, COGNOME, EMAIL, PASSWORD, AFFILIAZIONE, RUOLO) VALUES(?, ?, ?, ?, ?, ?,'autore');";  [3]
            try(PreparedStatement stmt = conn.prepareStatement(sql);) {  [4]
                stmt.setString(1, a.getId().toString());  [5]
                stmt.setString(2, a.getName());  [6]
                stmt.setString(3, a.getLastName());  [7]
                stmt.setString(4, a.getEmail());  [8]
                stmt.setString(5, a.getPassword());  [9]
                stmt.setString(6, a.getAffiliation());  [10]
                int ignore = stmt.executeUpdate();  [11]
            }
        } else if (user.getRole().equals("organizzatore")) {  [12]
            Organizer o = (Organizer) user;  [13]
            String sql = "INSERT INTO Utenti(ID, NOME, COGNOME, EMAIL, PASSWORD, AFFILIAZIONE, RUOLO) VALUES(?, ?, ? ,?, ?, ?,'organizzatore');";  [14]
            try(PreparedStatement stmt = conn.prepareStatement(sql);) { [ [15]
                stmt.setString(1, o.getId().toString());  [16]
                stmt.setString(2, o.getName());  [17]
                stmt.setString(3, o.getLastName());  [18]
                stmt.setString(4, o.getEmail());  [19]
                stmt.setString(5, o.getPassword());  [20]
                stmt.setString(6, o.getAffiliation());  [21]
                int ignore = stmt.executeUpdate();  [22]
            }
        }
    }
\end{lstlisting}
\paragraph{Control Flow Grap}
\begin{figure}[ht]
  \centering
  \includegraphics[width=\linewidth]{VisualParadigm/test_saveUser.png}
  \caption{Control Flow Grap di saveUser}
  \label{fig:cfg_di saveUser}
\end{figure}
\paragraph{Numero Ciclomatico}
\begin{itemize}
\item[.] numero di regioni chiuse del grafo +1 = 2 + 1= 3 
\item[.] numero di nodi predicati + 1 = 2 (1,12) +1= 3
\item[.] numero di archi - numero di nodi + 2 = 22 - 21 +2 = 3
\end{itemize}
\paragraph{Cammini}
\begin{itemize}
\item[Percorso 1:] 1 - 2 - 3 - 4 - 5 - 6 - 7 - 8 - 9 - 10 - 11
\item[Percorso 2:] 1 - 12 - 13 - 14 - 15 - 16 - 17 - 18 - 19 - 20 - 21 - 22
\item[Percorso 3:] 1 - 12
\end{itemize}
\paragraph{Casi di Test}
\begin{itemize}
\item[Casi n. 1:]
\begin{itemize}
\item[Obiettivo:] Verificare il percorso 1
\item[Condizioni:] Che l'user passato abbia come ruolo 
\begin{itemize}
\item[.] user.getRole().equals("autore") risulti true
\end{itemize}
\item[Input:] Che l'user in input sia un autore
\item[Risultato:] Che salvi l'user
\end{itemize}
\item[Casi n. 2:]
\begin{itemize}
\item[Obiettivo:] Verificare il percorso 2
\item[Condizioni:] Che l'user passato abbia come ruolo organizzatore
\begin{itemize}
\item[.] (user.getRole().equals("organizzatore") risulti true
\end{itemize}
\item[Input:] Che l'user in input sia un organizzatore
\item[Risultato:] Che salvi l'user
\end{itemize}
\item[Casi n. 3:]
\begin{itemize}
\item[Obiettivo:] Verificare il percorso 3
\item[Condizioni:] Che l'user passato abbia un ruolo diverso da organizzatore o autore; che l'user passato risulti null
\begin{itemize}
\item[.]  user.getRole().equals("autore") e (user.getRole().equals("organizzatore") risulti false
\end{itemize}
\item[Input:] Che l'user in input abbia un ruolo diverso da autore e organizzatore o che risulti null
\item[Risultato:] Che non salvi l'user
\end{itemize}
\end{itemize}

%%% Local Variables:
%%% mode: LaTeX
%%% TeX-master: "main"
%%% End: