\subsection{Analisi Nomi-Verbi}
\label{sec:analisi_nomi_verbi}
Si desidera sviluppare un sistema software per la gestione del
processo di sottomissione e revisione di articoli scientifici (anche
detti in gergo ‘paper’) nell’ambito di conferenze accademiche.
\bigskip

Il sistema coinvolge due tipologie di
\sethlcolor{WildStrawberry}\hl{utenti}:
\sethlcolor{Magenta}\hl{autori} e
\sethlcolor{Magenta}\hl{organizzatori}, ciascuna con funzionalità e
permessi specifici
\bigskip

\sethlcolor{Goldenrod}\hl{Ogni utente può registrarsi alla
  piattaforma} inserendo \sethlcolor{YellowGreen}\hl{nome},
\sethlcolor{YellowGreen}\hl{cognome},
\sethlcolor{YellowGreen}\hl{email} e
\sethlcolor{YellowGreen}\hl{affiliazione}. Al momento della
registrazione è possibile specificare il proprio
\sethlcolor{YellowGreen}\hl{ruolo}, scegliendo tra
\sethlcolor{Cerulean}\hl{autore} o
\sethlcolor{Cerulean}\hl{organizzatore}. Il sistema assegna un
\sethlcolor{YellowGreen}\hl{ID} univoco a ciascun utente registrato.
\bigskip

L’\sethlcolor{Cerulean}\hl{organizzatore} ha il compito di creare e
gestire una o più \sethlcolor{Cerulean}\hl{conferenze}. Mediante
un’apposita interfaccia grafica, \sethlcolor{Goldenrod}\hl{può creare
  una nuova conferenza} specificando
\sethlcolor{YellowGreen}\hl{titolo},
\sethlcolor{YellowGreen}\hl{descrizione} e una singola
\sethlcolor{YellowGreen}\hl{scadenza} per la chiusura delle
sottomissioni. Ogni conferenza creata sarà associata univocamente
all’organizzatore che l’ha generata. Una ulteriore sezione
dell’interfaccia grafica \sethlcolor{Goldenrod}\hl{consente
  all’organizzatore di visualizzare l’elenco completo dei}
\sethlcolor{Cerulean}\hl{paper} \sethlcolor{Goldenrod}\hl{ricevuti per
  ciascuna conferenza}. \sethlcolor{Goldenrod}\hl{Deve inoltre essere
  possibile accedere al dettaglio di ogni articolo e monitorarne lo
  stato}.  \bigskip

Ogni autore, una volta autenticato, dispone di una propria interfaccia
nella quale può \sethlcolor{Goldenrod}\hl{visualizzare l’elenco delle
  conferenze attive e sottomettere articoli a una conferenza attiva,
  compilando un modulo} che include
\sethlcolor{YellowGreen}\hl{titolo},
\sethlcolor{YellowGreen}\hl{abstract} (testo di al più 250 caratteri),
e una lista di \sethlcolor{YellowGreen}\hl{co-autori} (massimo tre in
totale). I co-autori devono essere già registrati al sistema e possono
essere selezionati dalla lista degli autori già registrati. Una volta
completata la sottomissione, l’articolo entra nello stato
“sottomesso”. \sethlcolor{Goldenrod}\hl{Gli autori, inoltre, possono
  visualizzare l’elenco dei propri articoli sottomessi, ciascuno con
  il relativo stato (sottomesso, in revisione)}.  \bigskip

L’organizzatore della conferenza, una volta superata la data di
scadenza delle sottomissioni, \sethlcolor{Goldenrod}\hl{deve assegnare
  tre} \sethlcolor{WildStrawberry}\hl{revisori}
\sethlcolor{Goldenrod}\hl{a ciascun articolo sottomesso}, che dovranno
effettuare la valutazione dell’articolo entro una data specificata
dall’organizzatore. I revisori \sethlcolor{Goldenrod}\hl{possono
  essere selezionati dalla lista di tutti gli autori già registrati
  nel sistema, oppure inserendo l’ID di un autore registrato}. Al
momento dell’assegnazione del revisore, il sistema deve impedire
conflitti di interesse, ossia un revisore non può essere autore del
paper che gli è stato assegnato . Quando ad un articolo sono stati
assegnati i tre revisori, l’articolo passa dallo stato “sottomesso”
allo stato “in revisione”.  \bigskip

\sethlcolor{Goldenrod}\hl{Gli organizzatori possono consultare in ogni
  momento lo stato aggregato della conferenza}, ossia il
\sethlcolor{YellowGreen}\hl{numero totale di articoli sottomessi}, il
\sethlcolor{YellowGreen}\hl{numero di articoli in revisione} e lo
\sethlcolor{YellowGreen}\hl{stato} di ogni articolo.  \bigskip

Il sistema deve garantire una gestione rigorosa dei permessi e delle
visibilità, mantenendo l’anonimato dei revisori nel processo di peer
review. \sethlcolor{Goldenrod}\hl{Inoltre, deve essere accessibile da
  dispositivi desktop e mobili, prevedere notifiche automatiche via
  email per le principali scadenze e offrire interfacce grafiche
  separate per autori e organizzatori}.  \bigskip

\textbf{LEGGENDA}:
\begin{itemize}
\item \sethlcolor{YellowGreen}\hl{ATTRIBUTO}
\item \sethlcolor{Cerulean}\hl{CLASSE}
\item \sethlcolor{WildStrawberry}\hl{ATTORE}
\item \sethlcolor{Magenta}\hl{CLASSE-ATTORE}
\item \sethlcolor{Goldenrod}\hl{FUNZIONALITÁ}
\end{itemize}
%%% Local Variables:
%%% mode: LaTeX
%%% TeX-master: "main"
%%% End:
