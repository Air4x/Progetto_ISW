\section{Pacchetto Database}
\label{sec:package_database}

\subsection{DBManager}
Classe che si occupa di impostare la connesione al database
\begin{description}
\item[DB_USER]  Il nome utente per il database
\item[DB_PASSWORD] La password per il database
\item[DB_URL]  L'url per il database
\end{description}
Tutti i parametri vengono ottenuti tramite \texttt{utility.PropertiesManager}
\subsubsection{\texttt{database.DBManager.DBManager()}}
Costruttore della classe \texttt{database.DBManager}, non è utilizzabile e di conseguenza
lancia una eccezzione se chiamato

\subsubsection{\texttt{database.DBManager.getConnection()}}
Metodo statico che permette di ottenere la connesione al database
\paragraph{Valore di ritorno}
\begin{itemize}
\item Un oggetto Connection che rappresenta la connesione al database
\end{itemize}


\subsection{UserDAO} Classe responsabile per le operazioni sulla
tabella utente del database
\begin{description}
\item[conn] La connesione al database, ottenuta da DBManager
\end{description}

\subsubsection{\texttt{database.UserDAO.UserDAO()}}
Costruttore di UserDAO, prova a creare una connesione con il database

\subsubsection{\texttt{database.UserDAO.getUserRoleByID()}}
Permette di ottenere il ruolo di un utente dato il suo id
\paragraph{Parametri}
\begin{itemize}
\item L'id dell'utente
\end{itemize}
\paragraph{Valore di ritorno}
\begin{itemize}
\item Il ruolo dell'utente sotto forma di stringa
\end{itemize}

\subsubsection{\texttt{database.UserDAO.getUserByID()}}
Permette di ottenere un utente dato il suo id
\paragraph{Parametri}
\begin{itemize}
\item L'id dell'utente
\end{itemize}
\paragraph{Valore di ritorno}
\begin{itemize}
\item Una istanza della classe
  \texttt{entity.Author}/\texttt{entity.Organizer} rappresentante
  l'utente
\end{itemize}

\subsubsection{\texttt{database.UserDAO.isUserPresentByID()}}
Predicato che verifica la presenza di un utente nel database
\paragraph{Parametri}
\begin{itemize}
\item L'id dell'utente
\end{itemize}
\paragraph{Valore di ritorno}
\begin{itemize}
\item Un valore booleano corrispondente alla presenza dell'utente
\end{itemize}

\subsubsection{\texttt{database.UserDAO.isUserPresentByEmail()}}
Predicato che verifica la presenza di un utente data la sua email
\paragraph{Parametri}
\begin{itemize}
\item L'email dell'utente
\end{itemize}
\paragraph{Valore di ritorno}
\begin{itemize}
\item Un valore booleano corrispondente alla presenza dell'utente
\end{itemize}

\subsubsection{\texttt{database.UserDAO.getUserByEmail()}}
Permette di ottenere un utente data la sua email
\paragraph{Parametri}
\begin{itemize}
\item L'email dell'utente
\end{itemize}
\paragraph{Valore di ritorno}
\begin{itemize}
\item La corrispondente istanza di \texttt{entity.Author} o di
  \texttt{entity.Organizer}
\end{itemize}

\subsubsection{\texttt{database.UserDAO.getAllAuthors()}} Permette
di ottenere una lista di tutti gli autori presenti nel database
\paragraph{Valore di ritorno}
\begin{itemize}
\item La lista di tutti gli autori nel database
\end{itemize}

\subsubsection{\texttt{database.UserDAO.saveUser()}} Permette di
inserire un nuovo utente all'interno del database
\paragraph{Parametri}
\begin{itemize}
\item L'utente da inserire
\end{itemize}


\subsection{ArticleDAO}
Classe responsabile per le operazioni sulla tabella articoli del database
\begin{description}
\item[conn] La connesione al database
\end{description}

\subsubsection{\texttt{database.ArticleDAO.ArticleDAO()}}
Costruttore, imposta la connesione al database

\subsubsection{\texttt{database.ArticleDAO.saveArticle()}}
Permette di aggiungere un articolo al database
\paragraph{Parametri}
\begin{itemize}
\item L'articolo da aggiungere
\end{itemize}
\paragraph{Valore di ritorno}

\subsubsection{\texttt{database.ArticleDAO.getArticleByID()}}
Permette di ottenere un articolo dato il suo id
\paragraph{Parametri}
\begin{itemize}
\item L'id dell'articolo
\end{itemize}
\paragraph{Valore di ritorno}
\begin{itemize}
\item Un istanza di Articolo rappresentante l'articolo ottenuto
\end{itemize}

\subsubsection{\texttt{database.ArticleDAO.getArticlesByAuthor()}}
Permette di ottenere tutti gli articoli scritti da un'utente
\paragraph{Parametri}
\begin{itemize}
\item L'id dell'autore
\end{itemize}
\paragraph{Valore di ritorno}
\begin{itemize}
\item La lista di articoli scritti dall'autore
\end{itemize}

\subsubsection{\texttt{database.ArticleDAP.updateArticleStatus()}}
Permette di aggiornare lo stato di un articolo
\paragraph{Parametri}
\begin{itemize}
\item L'id dell'articolo da aggiornare
\item Il nuovo stato
\end{itemize}


\subsubsection{ConferenceDAO}
Classe che si occupa delle operazioni sulla tabella conferenze nel database
\begin{description}
\item[conn] La connesione al database
\end{description}

\subsubsection{\texttt{database.ConferenceDAO.ConferenceDAO()}}
Costruttore, si occupa di impostare la connesione

\subsubsection{\texttt{database.ConferenceDAO.getConferenceByID()}}
Permette di ottenere una conferenza dato il suo di
\paragraph{Parametri}
\begin{itemize}
\item L'id della conferenza
\end{itemize}
\paragraph{Valore di ritorno}
\begin{itemize}
\item Un istanza di \texttt{entity.Conference} rappresentante la conferenza ottenuta
\end{itemize}

\subsubsection{\texttt{database.ConferenceDAO.saveConference()}}
\paragraph{Parametri}
\begin{itemize}
\item La conferenza da salvare
\end{itemize}

\subsubsection{\texttt{database.ConferenceDAO.getArticlesByConference()}}
Permette di ottenere la lista di tutti gli articoli sottomessi ad una conferenza
\paragraph{Parametri}
\begin{itemize}
\item L'id della conferenza
\end{itemize}
\paragraph{Valore di ritorno}
\begin{itemize}
\item La lista degli articoli sottomessi
\end{itemize}

\subsubsection{\texttt{database.ConferenceDAO.getAllConferences()}}
Permette di ottenere una lista di tutte le conferenze nel database
\paragraph{Valore di ritorno}
\begin{itemize}
\item La lista di tutte le conferenze
\end{itemize}

\subsubsection{\texttt{database.ConferenceDAO.getActiveConferences}}
Permette di ottenere la lista di tutte le conferenze attive
\paragraph{Valore di ritorno}
\begin{itemize}
\item La lista di tutte le conferenze attive
\end{itemize}
%%% Local Variables:
%%% mode: LaTeX
%%% TeX-master: "main"
%%% End:
