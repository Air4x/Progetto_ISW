\section{Pacchetto Utility}
\label{sec:package_utility}
Classe di utilità che implementa le funzioni per la creazione e
gestione dei codici identificativi usati nelle classi di dominio

Utilizza \texttt{java.util.UUID} per la creazione degli UUID veri e
propri, in particolare viene utilizzato per la creazione di UUIDv4 di
conseguenza la probabilità di una "convergenza" tra due ID è
abbastanza bassa da poter essere ignorata. Il tutto rafforzato dal
fatto che non ci sono state convergenze documentate utilizzando UUIDv4

\paragraph{Attributi}
\begin{description}
\item[id] L'id vero e proprio
\end{description}

\subsubsection{\texttt{utility.ID.ID()}}
Costruttore di ID, permette di creare un ID data la sua
rappresentazione come stringa
\paragraph{Parametri}
\begin{itemize}
\item L'id sotto forma di stringa
\end{itemize}

\subsubsection{\texttt{utility.ID.ID()}}
Costruttore di ID, crea un nuovo ID unico e universale

\subsubsection{\texttt{utility.ID.generate()}}
Metodo statico, viene utilizzato per generare un nuovo ID da parte
dei livello superiori
\paragraph{Valore di ritorno}
\begin{itemize}
\item Una nuova istanza di \texttt{utility.ID} contenente un nuovo UUID
\end{itemize}

\subsubsection{\texttt{utility.ID.getID()}}
Semplice getter per ottener l'id come \texttt{java.util.UUID}

\paragraph{Valore di ritorno}
\begin{itemize}
\item L'id come \texttt{java.util.UUID}
\end{itemize}

\subsubsection{\texttt{utility.ID.toString()}}
Override del metodo \texttt{toString()} di \texttt{java.util.UUID}
\paragraph{Valore di ritorno}
\begin{itemize}
\item L'id sotto forma di stringa
\end{itemize}

  
%%% Local Variables:
%%% mode: LaTeX    
%%% TeX-master: "mai
%%% End:         
