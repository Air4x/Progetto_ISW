\subsection{Traduzione classi ed associazioni}
\label{sec:prg_traduzione}

<<<<<<< HEAD
In tale \textbf{sezione} si va esplicitare il processo che dal modello concettuale ci porta al modello di progettazione, basato sull'architettura \textbf{BCED} (\textbf{Boundary}, \textbf{Controller}, \textbf{Entity}, \textbf{Database}).
Dove le scelte di progettazione sono state influenzzate da:
\begin{itemize}
\item Vincoli e le funzionalità espresse nei requisiti \ref{sec:analisi_nomi_verbi}
\item Rispettare i vicoli del modello \textbf{BCED}
\end{itemize}

\paragraph{Classi Entity}
\begin{description}
\item [User] Generalizzazione del concetto di utente, classe astratta, poiche nel sistema un user può essere solo o un \textbf{author} o un \textbf{organizer}. Caratterizatto dagli segueni attributi:
\begin{itemize}
\item ID: Che va a indetificare in maniera \textbf{univoca} un user
\item Name: nome dell'user
\item Lastname: cognome dell'user
\item Email: elemento identificativo per l'accesso al database per l'user
\item Password:  elemento identificativo per l'accesso al database per l'user 
\item Affiliazzione: organnizazione di appartenenza dell'user
\end{itemize}
\item [Author] Estende la classe \textbf{User}, ereditando tutti gli attributi e metodi. Caratterizzato dall'attributo:
\begin{itemize}
\item Final-Role: indica il ruolo dell'user (in questo caso \textbf{Autore})
\end{itemize}
\item [Organizer] Estende la classe \textbf{User}, ereditando tutti gli attributi e metodi. Caratterizzato dall'attributo:
\begin{itemize}
\item Final-Role: indica il ruolo dell'user (in questo caso \textbf{Organizzatore}) 
\end{itemize}
\item [Conference] rappresenta una conferenza creata da un \textbf{organizer}, caratterizzata dagli seguenti attributi:
\begin{itemize}
\item Titolo: Il nome della conferenzza
\item Descrizione: descrizione dell'argomento su cui è incetratta la conferenzza
\item Scadenza: data di scadenzza per la consegna degli articoli
\item Articoli: lista di riferimenti agli articoli sottomessi
\end{itemize}
\item [Article] rappresenta un articolo sottomesso, caratterizzato dagli seguenti attributi:
\begin{itemize}
 \item Titolo: Titolo dell'articolo
\item Abstract: Contenuto dell'articolo
\item Co-Autori: lista di riferimenti degli auto che hanno partecipato alla creazione dell'articolo
\end{itemize}
\end{description}

\paragraph{Associazioni}
\begin{description}
\item [Organizer->Conference] associazione di tipo: \textbf{1->* [uno a molti]} indica che uno stesso organizzatore può creare più conferenze
\item [Conference->Article] associazione di tipo: \textbf{1->* [uno a molti]} dove una confderenza può avere da uno a più articoli
\item [Author->Article] associazione di tipo: \textbf{*->* [molti a molti]} dove un articolo può essere scrito da uno o più autori mentre quest'ultimi possono scrivere più articoli.
\item [Article->Author] associazione di tipo: \textbf{*->3 [molti a molti]} con cui ad un aticolo si possono assegnare un messimo di 3 revisore attraverso ReviewDAO e ReviewController in particolar modo con:
\begin{itemize}
 \item assignReviewer(articleId, reviewerId)
\item getReviewersForArticle(articleId)
\end{itemize}
\end{description}

\paragraph{Controller}
\begin{description}
\item[TODO]
\end{description}

\paragraph{Data Access Layer}
\begin{description}
\item[TODO]
\end{description}

=======
>>>>>>> 907a50c7516ea0e29a9d6346c319ea8869f9d5d7
%%% Local Variables:
%%% mode: LaTeX
%%% TeX-master: "main"
%%% End:
