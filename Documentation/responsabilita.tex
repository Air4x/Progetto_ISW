\subsection{Responsabilita}
\label{sec:responsabilita}
\begin{tabular}{|p{5cm}|p{5cm}|}
\hline 
 \rowcolor{SkyBlue}
RESPONSABILITA & CLASSE\\
\hline
Registrazione & Sistema\\
\hline
Accesso & Sistema\\
\hline
Crea Conferenzza & Organizzatore\\
\hline
Sottometti Articolo & Autore\\
\hline
Assegna Revisori & Organizzatore\\
\hline
Notifica Scadenza & Sistema di Notifiche\\
\hline
Dispositivo di Accesso & Sistema\\
\hline
Gestione Permessi Visibilita & Sistema\\
\hline
Verifica Conflitti di Interesse & Sistema\\
\hline
\end{tabular}

\begin{itemize}
\item Registrazione e Accesso risultano responsabilita della classe Sistema, poiche sta a lui validare i dati e credenziali inserite.\\
\item Crea Conferenzza è responsabilita di Organizzatore, poiche è <<Creator>> di tale classe, andando poi a definire i vari parametri.\\
\item Assegna Revisori è responsabilita di Organizzatore, poiche è <<Creator>> di tale classe, andando ad assegna un revisore per ogni articolo.\\
\item Sottometti Articolo è responsabilita di Autore, poiche è <<Creator>> di tale classe, andando a sottomentere un dato articolo per una coonferenza.\\
\item Notifica Scadenza è responsabilita del Sitema di Notifiche andando ad automitizzando la comunicazione delle scadenze per gli utenti coinvolti.\\
\item Dispositivo di Accesso è responsabilita del Sistema permettere e controllare l'acceso degli Utenti da diversi dispositivi.\\
\item Gestione Permessi Visibilita è responsabilita del Sistema, anadando a gestire permessi.\\
\item Verifica Conflitti di Interesse è responsabilita del Sistema, poiche deve garatire l'assenza di conflitti tra articoli assegnati e rispettivi revisori.\\
\end{itemize}

%%% Local Variables:
%%% mode: LaTeX
%%% TeX-master: "main"
%%% End:
