\subsection{Scenari}
\label{sec:scenari}
\subsubsection{Registrazione}
\begin{tabular}{|p{1cm}|p{10cm}}
\hline 
\rowcolor{Skyblue}
Caso d'uso & Registrazione \\
\hline
Attore Primario & Utente\\
\hline
Attore Secondario & \\
\hline
Descrizione & Permette la registrazione di un utente\\
\hline
Sequenza Eventi&1.  Il caso d'uso inizia quando l'Utente inserisce Nome,Congome, Affiliazione, Email e Password\\
& 2. Il sistema verifica che l'email non sia utilizzata anche per altri\\ 
&3 Se il controllo è positivo e non si è verifcato alcun errrore\\
&3.1 il sistema informa all'utente la buona uscita dell'operazione e gli assegnerà un ID univoco\\
\hline
Post-Condizioni: & Il cliente è registrato nel sistema \\
\hline
Sequenza degli eventi alternativa & Al punto 3 il cliente immette una Email non compatibile o che sia già utilizzata restituendo un messaggio di errore\\
\hline
\end{tabular}


\subsubsection{Accesso}
\begin{tabular}{|p{1cm}|p{10cm}}
\hline 
\rowcolor{Skyblue}
Caso d'uso & Accesso \\
\hline
Attore Primario & Utente\\
\hline
Attore Secondario & \\
\hline
Descrizione & Permette l'accesso ad un utente al suo account\\
\hline
Pre-Condizioni& L'utente sia già registrato\\
\hline
Sequenza Eventi&1.  Il caso d'uso inizia quando l'Utente inserisce Email E Password\\
& 2. Il sistema verifica che l'email e  passoword coincidono con l'utente già registrato \\ 
&3 Se il controllo è positivo e non si è verifcato alcun errrore\\
&3.1 il sistema informa all'utente la buona uscita dell'operazione\\
\hline
Post-Condizioni: &il cliente ha l'accesso al suo account \\
\hline
Sequenza degli eventi alternativa & Al punto 3 il cliente immette una Email collegato ad alcun account  oppure che l'email e passoword non coincidono restituendo un messaggio di errore\\
\hline
\end{tabular}

\subsubsection{Crea Conferenza}
\begin{tabular}{|p{1cm}|p{10cm}}
\hline 
\rowcolor{Skyblue}
Caso d'uso & Crea Conferenza \\
\hline
Attore Primario & Organizzatore\\
\hline
Attore Secondario & Sistema Di notifica\\
\hline
Descrizione & L'organizzatore crea una conferenza \\
\hline
Sequenza Eventi&1. L'organizzatore crea una Conferenza inserendo il Titolo, una Descrizione dell'evento e una singola data di scadenza per le sottomissioni \\
& 2. Il sistema verifica che la data non sia passata o non sia valida \\ 
&3 Se il controllo è positivo e non si è verifcato alcun errrore\\
&3.1 il sistema informa all'utente la buona uscita dell'operazione\\
&3.2 Il sistema permette il monitoraggio dei vari articoli sottomessi per la conferenza\\
\hline
Post Condizioni: & L'organizzatore crea una conferenza a cui i vari autori possono sottomettere degli articoli \\
\hline
Sequenza degli eventi alternativa & Nel caso in cui la data di scadenza inserita non sia valida o che sia già passata invia un messaggio di errore \\
\hline
\end{tabular}

\subsubsection{SottomettiArticolo}
\begin{tabular}{|p{1cm}|p{10cm}}
\hline 
\rowcolor{Skyblue}
Caso d'uso & SottomettiArticolo \\
\hline
Attore Primario & Autore\\
\hline
Descrizione & L'autore sottomette un articolo per una determinata conferenza\\
\hline
Pre-Condizioni& Esiste la conferenza\\
\hline
Sequenza Eventi&1.  L'autore inserisce un articolo composto da un Titolo,Abstract e dai Co-Autori\\
& 2. Il sistema inserisce l'articolo e verifica che l'autore e i co-autori siano registrati al sistema \\ 
&3 Se il controllo è positivo e non si è verifcato alcun errrore\\
&3.1 il sistema informa all'utente la buona uscita dell'operazione\\
&3.2 il sistema inserisce l'articolo e permette agli autori di poter visualizzare lo stato dell'articolo\\
\hline
Post-Condizioni: &L'articolo è inserito e possibile per la verica degli articoli da parte dei Revisori\\
\hline
Sequenza degli eventi alternativa & Il sistema nel caso in cui i co-autori non sono stati registrati non procede con l'inserimento dell'articolo e restituisce un messaggio di errore\\
\hline
\end{tabular}

\subsubsection{AssegnaRevisori}
\begin{tabular}{|p{1cm}|p{10cm}}
\hline 
\rowcolor{Skyblue}
Caso d'uso & AssegnaRevisori \\
\hline
Attore Primario & Organizzatore\\
\hline
Attore Secondario & Sistema di notifica\\
\hline
Descrizione &L'organizzatore  assegna ad ogni articolo dei revisori per la revisione dell'articolo\\
\hline
Pre-Condizioni& L'Autore ha sottomesso l'articolo\\
\hline
Sequenza Eventi&1.  Il caso d'uso inizia allo scadere della data di revisione\\
&2. L'organizzatore assegna ad ogni articolo sottomesso tre revisori\\
&3 L'assegnazione può essere tramite ID oppure scegliendoli dalla lista di quelli registrati\\
\hline
Post-Condizioni: &L'articolo passa da uno stato di "sottomesso" a quello di "revisione" \\
\hline
Sequenza degli eventi alternativa \\
\hline
\end{tabular}ù

\subsubsection{NotificaScadenza}
\begin{tabular}{|p{1cm}|p{10cm}}
\hline 
\rowcolor{Skyblue}
Caso d'uso & NotificaScadenza \\
\hline
Attore Primario & Sistema di Notifica\\
\hline
Attore Secondario & Autore\\
\hline
Descrizione & Permette la notifica automatica via email per le principali scadenze\\
\hline
Pre-Condizioni& L'utente sia già registrato\\
\hline
Sequenza Eventi&1.  Il caso d'uso inizia quando la scadenza è imminente\\
& 2. Il sistema invia delle notifiche agli autori per le scadenze delle varie conferenze \\ 
&3. Tramite Email vengono notificati gli autori\\

\hline
Post-Condizioni: & \\
\hline
Sequenza degli eventi alternativa & \\
\hline
\end{tabular}