\chapter{Specifica Informale}
\label{sec:informal}
Si desidera sviluppare un sistema software per la gestione del
processo di sottomissione e revisione di articoli scientifici (anche
detti in gergo ‘paper’) nell’ambito di conferenze accademiche.
\bigskip

Il sistema coinvolge due tipologie di utenti: autori e organizzatori,
ciascuna con funzionalità e permessi specifici.
\bigskip

Ogni utente può registrarsi alla piattaforma inserendo nome, cognome,
email e affiliazione. Al momento della registrazione è possibile
specificare il proprio ruolo, scegliendo tra autore o
organizzatore. Il sistema assegna un ID univoco a ciascun utente
registrato.
\bigskip

L’organizzatore ha il compito di creare e gestire una o più
conferenze. Mediante un’apposita interfaccia grafica, può creare una
nuova conferenza specificando titolo, descrizione e una singola
scadenza per la chiusura delle sottomissioni. Ogni conferenza creata
sarà associata univocamente all’organizzatore che l’ha generata. Una
ulteriore sezione dell’interfaccia grafica consente all’organizzatore
di visualizzare l’elenco completo dei paper ricevuti per ciascuna
conferenza. Deve inoltre essere possibile accedere al dettaglio di
ogni articolo e monitorarne lo stato.
\bigskip

Ogni autore, una volta autenticato, dispone di una propria interfaccia
nella quale può visualizzare l’elenco delle conferenze attive e
sottomettere articoli a una conferenza attiva, compilando un modulo
che include titolo, abstract (testo di al più 250 caratteri), e una
lista di co-autori (massimo tre in totale). I co-autori devono essere
già registrati al sistema e possono essere selezionati dalla lista
degli autori già registrati. Una volta completata la sottomissione,
l’articolo entra nello stato “sottomesso”. Gli autori, inoltre,
possono visualizzare l’elenco dei propri articoli sottomessi, ciascuno
con il relativo stato (sottomesso, in revisione).
\bigskip

L’organizzatore della conferenza, una volta superata la data di
scadenza delle sottomissioni, deve assegnare tre revisori a ciascun
articolo sottomesso, che dovranno effettuare la valutazione
dell’articolo entro una data specificata dall’organizzatore. I
revisori possono essere selezionati dalla lista di tutti gli autori
già registrati nel sistema, oppure inserendo l’ID di un autore
registrato. Al momento dell’assegnazione del revisore, il sistema deve
impedire conflitti di interesse, ossia un revisore non può essere
autore del paper che gli è stato assegnato . Quando ad un articolo
sono stati assegnati i tre revisori, l’articolo passa dallo stato
“sottomesso” allo stato “in revisione”.
\bigskip

Gli organizzatori possono consultare in ogni momento lo stato
aggregato della conferenza, ossia il numero totale di articoli
sottomessi, il numero di articoli in revisione e lo stato di ogni
articolo.
\bigskip

Il sistema deve garantire una gestione rigorosa dei permessi e delle
visibilità, mantenendo l’anonimato dei revisori nel processo di peer
review. Inoltre, deve essere accessibile da dispositivi desktop e
mobili, prevedere notifiche automatiche via email per le principali
scadenze e offrire interfacce grafiche separate per autori e
organizzatori.

%%% Local Variables:
%%% mode: LaTeX
%%% TeX-master: "main"
%%% End:
